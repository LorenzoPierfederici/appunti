%-*-coding: utf-8;-*-
\documentclass[italian,a4paper,hidelinks]{scrartcl}
\usepackage{amsmath,amssymb,amsthm,thmtools}
\usepackage{eucal,babel,a4}
\usepackage[nochapters]{classicthesis}
\usepackage[utf8]{inputenc}

\newcommand{\RR}{{\mathbb R}}
\newcommand{\C}{{\mathcal C}}
\newcommand{\defeq}{=}

\def\Xint#1{\mathchoice
{\XXint\displaystyle\textstyle{#1}}%
{\XXint\textstyle\scriptstyle{#1}}%
{\XXint\scriptstyle\scriptscriptstyle{#1}}%
{\XXint\scriptscriptstyle\scriptscriptstyle{#1}}%
\!\int}
\def\XXint#1#2#3{{\setbox0=\hbox{$#1{#2#3}{\int}$ }
\vcenter{\hbox{$#2#3$ }}\kern-.6\wd0}}
\def\dashint{\Xint-}

\declaretheoremstyle[
spaceabove=6pt, spacebelow=6pt,
headfont=\normalfont\itshape,
notefont=\mdseries, notebraces={(}{)},
bodyfont=\normalfont,
postheadspace=1em,
qed=,
shaded={rulecolor=pink!30,rulewidth=1pt,bgcolor=pink!10}
]{mystyle}

\declaretheorem[numberwithin=section,name=Teorema]{theorem}
\declaretheorem[sibling=theorem,name=Lemma]{lemma}
\declaretheorem[style=mystyle,sibling=theorem,name=Esercizio]{exercise}
\declaretheorem[style=mystyle,sibling=theorem,name=Esempio]{example}


\title{Curve e integrali curvilinei}
\author{E. Paolini}
\date{13 ottobre 2014}

\begin{document}
\maketitle

\section{Curve parametrizzate}

Una \emph{curva parametrizzata}
\marginpar{curva parametrizzata\\estremi\\curve chiuse\\supporto}
è una funzione $\gamma\colon [a,b]\to
\RR^n$. Al variare di $t$ nell'intervallo $[a,b]$ (con $a<b$) il punto
$\gamma(t)$ descrive una traiettoria nello spazio $\RR^n$. I punti
$\gamma(a)$ e $\gamma(b)$ si chiamano \emph{estremi} della
curva. Se $\gamma(a)=\gamma(b)$ la curva si dice essere \emph{chiusa}.
L'immagine della funzione $\gamma$ si indica con $[\gamma]$
\[
  [\gamma] \defeq f([a,b]) \subset \RR^n
\]
e si chiama \emph{supporto della curva}.

\begin{example}[il segmento]
Dati due punti $\mathbf p, \mathbf q\in \RR^n$ si può definire il
segmento di retta che congiunge $\mathbf p$ con $\mathbf q$ mediante la funzione
$\gamma\colon [0,1] \to \RR^n$ definita da
\[
  \gamma(t) = t \mathbf q + (1-t)\mathbf p
\]
Gli estremi della curva sono $\gamma(0) = \mathbf p$ e $\gamma(1)=
\mathbf q$. Il vettore
\[
  \gamma'(t) = \mathbf q- \mathbf p
\]
rappresenta la velocità (in questo caso costante) con cui il punto
$\gamma(t)$ si muove lungo la curva.
\end{example}

Se la funzione $\gamma$ è di classe $\C^1$ e se $\lvert
\gamma'(t)\rvert \neq 0$ per ogni $t \in [a,b]$ diremo che la curva $\gamma$ è
\marginpar{curva regolare}
\emph{regolare}. In tal caso diremo che la \emph{direzione tangente}
\marginpar{direzione tangente}
alla
curva $\gamma$ nel punto $\gamma(t)$ è il versore
\[
    \frac{\gamma'(t)}{\lvert \gamma'(t)\rvert}.
\]
Questa definizione è giustificata dal fatto che la direzione
\emph{secante} tra i punti $\gamma(t)$ e $\gamma(t+h)$ (con $h>0$)
è data da
\[
     \frac{\gamma(t+h)-\gamma(t)}{\lvert \gamma(t+h) -
       \gamma(t)\rvert}
     =
     \frac{\frac{\gamma(t+h)-\gamma(t)}{h}}{\left|\frac{\gamma(t+h)-\gamma(t)}{h}\right|}
\]
e per $h\to 0^+$ tende in effetti alla direzione tangente (mentre per $h\to 0^-$ si otterrà la direzione opposta).

\section{lunghezza e integrale curvilineo}
Supponiamo che $\gamma$ sia di classe $\C^1$.
Se $\gamma'(t)$ rappresenta la velocità (istantanea, al tempo $t$) con
cui il punto $\gamma(t)$ si muove lungo la sua traiettoria, la
velocità \emph{scalare} sarà data da $v(t) = \lvert \gamma'(t)
\rvert$.
\marginpar{lunghezza}
Dunque definiamo la \emph{lunghezza della curva} $\gamma$ tramite
l'integrale della velocità scalare:
\[
\ell(\gamma) \defeq \int_a^b \lvert \gamma'(t)\rvert \, dt.
\]
Se poi $f\colon [\gamma]\subset \RR^n \to \RR$ è una funzione definita
sui punti del supporto di $\gamma$ e tale che
$f\circ \gamma\colon [a,b]\to \RR$ sia integrabile, si può definire
\marginpar{integrale curvilineo}
\emph{l'integrale curvilineo} di $f$ su $\gamma$ mediante la formula
\[
\int_\gamma f\, ds \defeq \int_a^b f(\gamma(t)) \lvert
\gamma'(t)\rvert\, dt.
\]

\begin{example}[la cicloide]
Si consideri il moto di un punto $\gamma$ che si trova su una
circonferenza di raggio $R$ che rotola senza strisciare. Usando un
sistema di coordinate in cui $y$ è la verticale e $x$ è l'ascissa
lungo la retta di rotolamento.

Se la velocità angolare della ruota è $\omega=1$ le coordinate del punto
$\gamma(t)$ si possono ottenere mediante la composizione di un moto
rotatorio (diciamo in senso orario) e di una traslazione a velocità
costante nella direzione dell'asse $x$. Si potranno quindi ottenere
queste equazioni per un \emph{giro} completo della ruota:
\begin{gather*}
\gamma\colon [0,2\pi] \to \RR^2,\qquad \gamma(t)=(x(t), y(t))\\
\begin{cases}
x(t) = R t - R \sin t \\
y(t) = R - R \cos t.
\end{cases}
\end{gather*}
Facendo le derivate si ottiene:
\[
\begin{cases}
x'(t) = R - R \omega \cos t \\
y'(t) = R \sin t
\end{cases}
\]
da cui
\begin{align*}
 \lvert\gamma'(t)\rvert & = \sqrt{(x'(t))^2 + (y'(t))^2}
 =  R \sqrt{ (1-\cos t)^2 + \sin^2 t} \\
 & =  R \sqrt{2 - 2 \cos t}
 =  R \sqrt{2 - 2 (1 - 2\sin^2 (t/2))} \\
 & =  R \sqrt{4 \sin^2(t/2)} = 2\omega R \lvert \sin(t/2)\rvert.
\end{align*}

Da cui la lunghezza della curva percorsa è data da
\[
\ell(\gamma) = \int_0^{2\pi} \lvert \gamma'(t)\rvert \, dt
 = 2 \omega R \int_0^{2\pi} \sin(t/2)\, dt = -4 [ \cos(t/2)]_0^{2\pi}
 = 8.
\]
\end{example}

\section{Riparametrizzazioni}

Se $\gamma\colon [a,b]\to \RR^n$ è una curva e $p\colon [c,d]\to
[a,b]$ è una funzione continua e bigettiva, componendo $\gamma$ con
$p$ si ottiene un'altra \emph{parametrizzazione}
\marginpar{riparametrizzazione}
(cioé una
\emph{riparametrizzazione}) $\sigma$ della curva
$\gamma$:
\[
\sigma(t) = \gamma(p(t)).
\]

Osserviamo che se $p\colon [c,d] \to [a,b]$ è continua e iniettiva,
necessariamente $p$ è strettamente monotona (altrimenti tramite il
teorema dei valori intermedi si otterrebbe un assurdo) e quindi ci
sono due sole possibilità: o $p$ è strettamente crescente oppure $p$ è
strettamente decrescente. Nel primo caso perché $p$ sia anche
surgettiva si deve avere $p(c)=a$ e $p(d)=b$. Nel secondo caso (quando
$p$ è decrescente) si avrà $p(c) = b$, $p(d) = a$. Inoltre in ogni
caso la funzione inversa $p^{-1}\colon [a,b] \to [c,d]$ sarà anch'essa
bigettiva e continua.

Date le curve $\gamma\colon [a,b]\to \RR^n$ e $\sigma\colon[c,d] \to
\RR^n$ diremo che $\gamma$ e $\sigma$ sono \emph{equivalenti} se
esiste una funzione continua e bigettiva $p\colon[c,d]\to [a,b]$ tale
che $\sigma(t) = \gamma(p(t))$. Se $p$ è strettamente crescente diremo
inoltre che $\gamma$ e $\sigma$ hanno la
\marginpar{orientazione}
\emph{stessa orientazione},
in caso contrario diremo che hanno \emph{orientazione opposta}.

E' facile verificare che due curve equivalenti hanno lo stesso supporto $[\gamma] =
[\sigma]$.

\begin{lemma}
Se $\gamma$ e $\sigma$ sono equivalenti e regolari, allora la funzione
$p$ tale che $\gamma(t) = \sigma(p(t))$ è di classe $\C^1$,
invertibile, con inversa di classe $\C^1$.
\end{lemma}

\begin{proof}
Fissiamo un punto $t\in [c,d]$. Visto che $\lvert \gamma'(p(t))\rvert
\neq 0$. Questo significa che posto $\gamma(t) =
(\gamma_1(t),\dots,\gamma_n(t))$ c'è almeno una componente
$\gamma_k(t)$ tale che $\gamma_k'(t)\neq 0$. Supponiamo per semplicità
$\gamma_k'(t)>0$. Essendo $\gamma_k\in \C^1$ si ha che $\gamma_k'$ è
continua e quindi per la permanenza del segno sarà $\gamma_k'(\tau)
>0$
per ogni $\tau$ in un opportuno intorno di $t$. Dunque $\gamma_k$ è
invertibile, con inversa derivabile, in un intorno di $t$ ed essendo:
\[
  \sigma_k(t) = \gamma_k(p(t))
\]
si può scrivere
\[
  p(t) = \gamma_k^{-1}(\sigma_k(t)).
\]
da cui si ottiene che $p(t)$ è derivabile e la sua derivata è
continua. Lo stesso vale per la funzione inversa $p^{-1}(t)$
scambiando $\gamma$ con $\sigma$.
\end{proof}

\begin{theorem}[invarianza dell'integrale curvilineo]
Se $\sigma$ e $\gamma$ sono curve regolari equivalenti, e sia $f$ una
funzione continua definita sul loro supporto. Allora
\[
 \int_\sigma f \, ds = \int_\gamma f \, ds.
\]
\end{theorem}

\begin{proof}
Sia $p(t)$ la riparametrizzazione tale che
$\sigma(t) =
\gamma(p(t))$. Per il lemma precedente sappiamo che $p$ è una funzione
di classe $\C^1$ con inversa di classe $\C^1$. Supponiamo che sia
$p'(t)>0$ (cioè le curve hanno la stessa orientazione). Allora si ha
\begin{align*}
\int_\gamma f\, ds = \int_a^b f(\gamma(\tau)) \lvert \gamma'(\tau)\rvert\, d\tau
= \int_c^d f(\gamma(p(t)) \lvert \gamma'(p(t))\rvert\,
p'(t)\, dt
\end{align*}
dove abbiamo applicato il cambio di variabili $\tau = p(t)$, da cui
$d\tau = p'(t) dt$ e $p(c)=a$ e $p(d)=b$ (gli estremi vengono
mantenuti nello stesso ordine se l'orientazione è la stessa).

Osserviamo ora che $\gamma(p(t)) = \sigma(t)$ e derivando questa
identità osserviamo anche che $\gamma'(p(t))p'(t) = \sigma'(t)$ da cui
(ricordando che $\lvert p'(t)\rvert = p'(t)$) si ottiene
\[
  \int_\gamma f\, ds = \int_c^d f(\sigma(t)) \lvert \sigma'(t)\rvert\,
  dt
  = \int_\sigma f\, ds.
\]

Dunque il teorema è dimostrato nel caso in cui $\gamma$ e $\sigma$
abbiano orientazione concorde. Nel caso di orientazione opposta
osserviamo che si avrà $\lvert p'(t)\rvert = - p'(t)$ ma gli estremi
dell'integrale saranno scambiati: $p(c)=b$, $p(d)=a$... dunque alla
fine il risultato non cambia.
\end{proof}

\section{Parametrizzazione per lunghezza d'arco}

Se $\gamma(t)$ rappresenta la posizione di un punto al tempo $t$, la
quantità
$$
s(t) =  \ell(\gamma_{|[a,t]}) = \int_0^t |\gamma'(\tau)|\, d\tau
$$
\marginpar{lunghezza d'arco}
si chiama \emph{lunghezza d'arco} e
rappresenta la distanza percorsa dal punto $\gamma(t)$ nell'intervallo
di tempo $[a,t]$. In altri termini $s(t)$ rappresenta la lunghezza del
tratto di curva compreso tra $\gamma(a)$ e $\gamma(t)$.

Se $\gamma$ è
una curva regolare possiamo osservare che (per il teorema fondamentale
del calcolo integrale)
\[
s'(t) = |\gamma'(t)| > 0
\]
dunque posto $L=\ell(\gamma)=s(b)$ osserviamo che
$s\colon[a,b]\to[0,L]$ è bigettiva e continua e $\eta(s) =
\gamma(s^{-1}(s))$ è una riparametrizzazione. Osserviamo qui l'abuso
di notazione avendo utilizzato la lettera $s$ sia per indicare la
funzione $s(t)$ che per indicare la variabile $s=s(t)$.

Diremo che una curva regolare $\gamma\colon [0,L]\to \RR^n$
è parametrizzata \emph{per lunghezza
  d'arco} se si ha $\lvert \gamma'(t)\rvert = 1$ per ogni
$t\in[0,L]$.
In tal caso si avrà $s(t)=t$, infatti essendo unitaria la velocità, i
parametri tempo e spazio coincidono. Inoltre $L=\ell(\gamma)$ risulta
essere la lunghezza totale della curva.

Spesso la variabile $s$ è riservata alle parametrizzazioni per
lunghezza d'arco. Questo spiega
il simbolo $ds$ utilizzato nella notazione degli
integrali curvilinei, infatti
quando $\gamma(s)$ è parametrizzata per lunghezza d'arco
si ha
\[
  \int_\gamma f\, ds = \int_0^L f(\gamma(s))\lvert \gamma'(s)\rvert\,
  ds
 = \int_0^L f(\gamma(s))\, ds
\]

Se $\gamma(t)$ con $t\in[a,b]$ è una curva con una parametrizzazione qualunque, si può
determinare la riparametrizzazione che ci porta ad una curva
$\sigma(s)$ parametrizzata per lunghezza d'arco. Si tratta di
calcolare
\[
  s(t) = \int_a^t \lvert\gamma'(\tau)\rvert\, d\tau
\]
e quindi definire $\sigma(s) \defeq \gamma(s^{-1}(s))$. La funzione
$s(t)$, infatti, ci permette di convertire il parametro $t$ (il tempo)
nel parametro $s$ (lo spazio percorso, o lunghezza d'arco). Per ottenere una
parametrizzazione rispetto alla lunghezza d'arco è quindi necessario
invertire tale funzione. Verifichiamo che $\sigma(s)=\gamma(s^{-1}(t))$ è parametrizzata
per lunghezza d'arco. Ricordando che
$s'(t) = \lvert \gamma'(t)\rvert$ si ha:
\[
\lvert \sigma'(s) \rvert = \lvert \gamma'(s^{-1}(s))
(s^{-1})'(s)\rvert  = \frac{\lvert \gamma'(s^{-1}(s))\rvert}{\lvert
  s'(s)\rvert}
= 1.
\]

Osserviamo ora che avendo posto $s=s(t)$ si potrebbe utilizzare la
notazione $t=t(s)$ per indicare la funzione inversa di $s(t)$.
Questo è coerente con la
notazione di Leibniz per le derivate:
\[
  \frac{ds}{dt} = \frac{1}{\frac{dt}{ds}}
\]
Allo stesso modo si userà di frequente lo stesso simbolo per indicare
$\gamma$ e $\sigma$:
\[
  \gamma(s) \defeq \gamma(t(s))
\]
da cui se $\lvert d\gamma/ds \rvert = 1$ si ha, formalmente:
\[
  ds = \left| \frac{d\gamma}{ds}\right| ds =
  \left|\frac{d\gamma}{dt}\right| \left|\frac{dt}{ds}\right| ds =
  \lvert \gamma'(t)\rvert dt.
\]

\begin{exercise}[baricentro di un arco di circonferenza]
Ci poniamo l'obiettivo di calcolare il baricentro di un arco di
circonferenza di raggio $R$ e ampiezza $\alpha$.
Sia $\gamma(t) = (x(t),y(t))$ con $t\in [0,\alpha]$
\[
\begin{cases}
x(t) = R \cos t \\
y(t) = R \sin t
\end{cases}
\]
Le coordinate del baricentro $(\bar x, \bar y)$ sono date da
\[
\begin{cases}
\bar x = \dashint_\gamma x(s)\, ds \defeq \frac{\int_\gamma x(s)\, ds}{\ell(\gamma)}\\
\bar y = \dashint_\gamma y(s)\, ds \defeq \frac{\int_\gamma y(s)\, ds}{\ell(\gamma)}
\end{cases}
\]
Osserviamo ora che
\[
  \lvert \gamma'(t)\rvert = \sqrt{ x'(t)^2 + y'(t)^2}
  = R
\]
da cui si ricava $ds = R dt$ e quindi
$\ell(\gamma) = \int_0^\alpha R\, dt = \alpha R$.

Allora possiamo calcolare le coordinate $(\bar x,\bar y)$ del baricentro:
\begin{align*}
\bar x & = \dashint_\gamma x\, ds =
\frac{1}{\alpha R} \int (R\cos t) R\, dt \\
& =
\frac{R}{\alpha}[\sin t]_0^\alpha
= R \frac{\sin \alpha}{\alpha}.\\
\bar y &= \dashint_\gamma y\, ds =
\frac{1}{\alpha R} \int (R\sin t) R\, dt \\
& = \frac{R}{\alpha} [-\cos t]_0^\alpha
 = R\frac{1-\cos \alpha}{\alpha}.
\end{align*}
\end{exercise}

\section{Registro modifiche}
\begin{itemize}
\item[2014-10-13] Prima stesura.
\item[2014-10-17] Correzioni/piccole modifiche.
\end{itemize}
\end{document}
