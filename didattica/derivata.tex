%-*-coding: utf-8;-*-
\documentclass[italian,a4paper]{scrartcl}
\usepackage{amsmath,amssymb,amsthm,thmtools}
\usepackage{eucal,babel,a4}
\usepackage[nochapters]{classicthesis}
\usepackage[utf8]{inputenc}
\usepackage{verbatim} % for the comment environment
\usepackage{graphicx}
\usepackage{hyperref}

\newcommand{\RR}{{\mathbb R}}
\newcommand{\NN}{{\mathbb N}}
\newcommand{\ZZ}{{\mathbb Z}}
\newcommand{\QQ}{{\mathbb Q}}
\newcommand{\C}{{\mathcal C}}
\newcommand{\eps}{\varepsilon}
\newcommand{\defeq}{=}
\renewcommand{\vec}{\mathbf}
\renewcommand{\div}{\mathrm{div}}
\newcommand{\rot}{\mathbf{rot}\,}
\newcommand{\vecnabla}{\mathbf{\nabla}}
\newcommand{\tr}{\mathrm{tr}}

\def\Xint#1{\mathchoice
{\XXint\displaystyle\textstyle{#1}}%
{\XXint\textstyle\scriptstyle{#1}}%
{\XXint\scriptstyle\scriptscriptstyle{#1}}%
{\XXint\scriptscriptstyle\scriptscriptstyle{#1}}%
\!\int}
\def\XXint#1#2#3{{\setbox0=\hbox{$#1{#2#3}{\int}$ }
\vcenter{\hbox{$#2#3$ }}\kern-.6\wd0}}
\def\dashint{\Xint-}

\declaretheoremstyle[
spaceabove=6pt, spacebelow=6pt,
headfont=\normalfont\itshape,
notefont=\mdseries, notebraces={(}{)},
bodyfont=\normalfont,
postheadspace=1em,
qed=,
shaded={rulecolor=pink!30,rulewidth=1pt,bgcolor=pink!10}
]{mystyle}

\declaretheorem[name=Teorema]{theorem}
\declaretheorem[name=Lemma]{lemma}
\declaretheorem[name=Definizione]{definition}
\declaretheorem[style=mystyle,name=Esercizio]{exercise}
\declaretheorem[style=mystyle,name=Esempio]{example}

\title{Linearizzazione del campo gravitazionale\\(introduzione alla derivata)}
\author{Analisi Matematica 1\\Fisica a.a. 2015-2016}

\begin{document}
\maketitle

Il campo gravitazionale generato dalla terra nello spazio, in un punto
a distanza $r$ dal suo centro è dato da:
\[
U(r) = -\frac{GM}{r}
\]
dove $M$ è la massa della terra e $G$ è la costante di gravitazione universale.
La funzione $U(r)$ non è affatto lineare. Se però consideriamo il
campo gravitazionale per i punti in prossimità della
superficie terrestre, ci aspettiamo un comportamento approssimativamente
lineare. Proviamo a esplicitare questa idea.

Supponiamo di trovarci ad altezza $h$ dalla superficie terrestre. Ci
troveremo allora a distanza $R+h$ dal centro della terra. Si avrà allora:
\[
  U(R+h) = -GM\frac{1}{R+h}. 
\]
Osserviamo ora che si ha
\begin{align*}
  \frac{1}{R+h}
  & = \frac{1}{R} + \frac{1}{R+h} - \frac{1}{R}
   = \frac 1 R + \frac{R-(R+h)}{R(R+h)} \\
  & = \frac 1 R - \frac{h}{R(R+h)} \\
  & = \frac 1 R - \frac{h}{R^2} - \frac{h}{R(R+h)} + \frac{h}{R^2} \\
  & = \frac 1 R - \frac{h}{R^2} - \frac{hR - h(R + h)}{R^2(R+h)} \\
  & = - \frac{h}{R^2} + \frac 1 R + \frac{h^2}{R^2(R+h)}.
\end{align*}
Dunque si avrà
\begin{align*}
  U(R+h) &= \frac{GM}{R^2} h - \frac{GM}{R} + \omega(h)\\
  &= g h + C + \omega(h)
\end{align*}
dove $g = GM/R^2$, $C$ è una costante (irrilevante perché il
potenziale può essere definito a meno di una costante) e
$\omega(h)$ è una funzione con la proprietà $\omega(h)/h\to 0$
per $h\to 0$. Dunque se $h$ è molto piccolo rispetto a $R$, il termine
$\omega(h)$ è trascurabile rispetto al termine $gh$ (anche se entrambi
tendono a zero per $h\to 0$). Questo giustifica
l'utilizzo della formula
semplificata:
\[
U_0(h) = gh
\]
da cui l'energia potenziale $E = mgh$
se abbiamo una massa $m$ ad una altezza $h$ sulla superficie terrestre.

Una volta introdotte le derivate vedremo che quello che abbiamo
determinato è la formula:
\[
U(R+h) = U(R) + U'(R) h + \omega(h).
\]
\end{document}

