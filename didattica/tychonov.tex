\documentclass{article}
\usepackage[italian]{babel}
\usepackage[latin1]{inputenc}
\usepackage{amsmath,amssymb,amsthm}
\usepackage{eucal}

\newtheorem{theorem}{Teorema}
\newtheorem{proposition}[theorem]{Proposizione}
\theoremstyle{definition}
\newtheorem{definition}[theorem]{Definizione}

\newtheorem{lemma}[theorem]{Lemma}
\newtheorem{corollary}[theorem]{Corollario}

\theoremstyle{remark}
\newtheorem{remark}[theorem]{Osservazione}
\newtheorem{example}[theorem]{Esempio}

\newcommand{\defeq}{:=}
\newcommand{\F}{\mathcal F}
\newcommand{\M}{\mathcal M}

\title{Il teorema di Tychonov}
\author{Emanuele Paolini}


\begin{document}
\maketitle

\begin{definition}
Sia $X$ uno spazio topologico. $X$ si dice \emph{compatto} se da ogni
ricoprimento aperto � possibile estrarre un sottoricoprimento finito:
\end{definition}

\begin{definition}
Uno spazio topologico $X$ si dice $T_2$ o di \emph{Hausdorff} se dati
comunque $x,y\in X$ punti distinti, esistono un intorno $U$ di $x$ e
un intorno $V$ di $y$, disgiunti.
\end{definition}

\begin{proposition}\label{prop1}
Sia $X$ spazio topologico $T_2$, $K\subset X$ compatto e $x_0 \in
X\setminus K$. 
Allora esistono due aperti disgiunti $U,V$ tali che $K\subset U$ e $x_0\in V$.
\end{proposition}
\begin{proof}
Visto che $X$ � $T_2$ per ogni $x\in K$ esistono un aperto $U_x\ni x$
e un aperto $V_x\ni x_0$ disgiunti. La famiglia $\{U_x\colon x \in
K\}$ risulta essere un ricoprimento aperto di $K$ dal quale, essendo
$K$ compatto, posso estrarre un ricoprimento finito: $\{U_{x_i}\colon
x_i \in K, i=1,\dots, N\}$. 
Allora posto
\[
  U \defeq \bigcup_{i=1}^N U_{x_i}, \quad V= \bigcap_{i=1}^N V_{x_i}
\]
risulta che $U,V$ sono aperti disgiunti con $U\supset K$ e $x_0 \in V$.
\end{proof}

\begin{proposition}\label{prop2}
Sia $X$ uno spazio topologico compatto e $T_2$. Allora ogni intorno di
$x_0\in X$ contiene un intorno chiuso di $x_0$.
\end{proposition}
\begin{proof}
Sia $W$ un generico intorno aperto di $x_0$. Allora $X\setminus W$ �
compatto, perci� esistono per la Proposizione~\ref{prop1} due
aperti disgiunti $U,V$ con $U\supset X\setminus W$ e $V\ni x_0$.
Dunque $X\setminus U$ � un intorno chiuso di $x_0$ contenuto in $W$.
\end{proof}

\begin{definition}
Una famiglia $\F$ di insiemi si dice aver la \emph{propriet�
  dell'intersezione finita} (PIF) se per ogni $\F_0\subset \F$ finito si
ha
\[
  \bigcap \F_0 \defeq \bigcap_{A\in \F_0} A \neq \emptyset.
\]
\end{definition}

Osserviamo che uno spazio topologico $X$ � compatto se e solo se ogni
famiglia $\F$ di chiusi di $X$ con la propriet� PIF ha intersezione
non vuota: $\bigcap \F \neq \emptyset$. 

Infatti si tratta semplicemente di osservare che il passaggio di ogni
insieme al suo complementare mette in corrispondenza i seguenti
concetti
\begin{center}
\begin{tabular}{rcl}
aperto &$\leftrightarrow$& chiuso \\
ricoprimento dello spazio &$\leftrightarrow$& intersezione vuota\\
sottoricoprimento finito &$\leftrightarrow$& sottofamiglia finita con
intersezione vuota
\end{tabular}
\end{center}
dunque passando ai complementari e invertendo le implicazioni si ottiene
\begin{align*}
\text{$\forall \mathcal R$ famiglia di aperti:}
\quad  \bigcup \mathcal R=X &\implies \exists \mathcal R_0\subset
\mathcal R \text{
  finito } \bigcup \mathcal R_0=X\\
&\ \ \Updownarrow \\
\text{$\forall \F$ famiglia di chiusi:}
\quad \bigcap \F \neq \emptyset &\impliedby \forall \F_0 \subset \F
\text{ finito } \bigcap \F_0 \neq \emptyset
\end{align*}
cio� 
\begin{remark}\label{oss1}
Uno spazio $X$ � compatto se e solo se ogni famiglia di chiusi
con la propriet� PIF ha intersezione non vuota.
\end{remark}

\begin{definition}
Diciamo che $\M$ � una famiglia \emph{massimale} di chiusi con la
PIF se non esiste alcuna famiglia di chiusi $\M'\supset \M$
con la PIF.
\end{definition}

\begin{proposition}\label{prop3}
Se $\F$ � una famiglia di chiusi con la PIF esiste sempre una famiglia
di chiusi $\M\supset \F$ massimale, con la PIF.
\end{proposition}
\begin{proof}
Si tratta di applicare il Lemma di Zorn. Sulle famiglie con la PIF di chiusi contenenti $\F$ 
consideriamo l'ordinamento dato dall'inclusione. Vogliamo allora
dimostrare che ogni catena $\F_\alpha$ di famiglie con la PIF pu�
essere maggiorata con la sua unione $\F\defeq \bigcup_\alpha
\F_\alpha$. $\F$ � chiaramente una famiglia di insiemi chiusi.
Osserviamo che scelta comunque una sottofamiglia $\F_0\subset \F$ finita, 
� possibile trovare $\alpha$ tale che $\F_0\subset \F_\alpha$ in
quanto ogni elemento $C_i$ di $\F_0$ sta in un qualche $\F_{\alpha_i}$, e visto
che ne abbiamo solo un numero finito esister� nella catena 
un $\alpha\ge \alpha_i$
per ogni $i$.
Dunque siccome $\F_\alpha$ ha la PIF, si conclude che $\bigcap \F_0\neq
\emptyset$.
Dunque $\F$ ha la PIF, e quindi � un maggiorante ammissibile della
catena $\F_\alpha$.

Il Lemma di Zorn ci assicura allora l'esistenza di un elemento
massimale $\M$ ovvero una famiglia massimale di chiusi con la propriet� $PIF$
contenente $\F$.
\end{proof}

\begin{proposition}\label{prop4}
Se $\M$ � una famiglia massimale di chiusi con la PIF allora
\begin{enumerate}
\item[(i)] se $A,B\in \M$ anche $A \cap B \in \M$;
\item[(ii)] se $A\cap B \neq \emptyset$ per ogni $B \in \M$, allora $A\in \M$.
\end{enumerate}
\end{proposition}
Per il primo punto basta mostrare che $\M'\defeq \M \cup \{A\cap B\}$
ha la PIF. Sia dunque $\F_0$ una sottofamiglia finita di $\M'$. Se
$\F_0\subset \M$ chiaramente $\F_0$ avr� la PIF. In caso contrario
$\F_0$ sar� della forma
\[
 \F_0 = \{ A_1, A_2, \dots, A_N, A\cap B\}.
\]
Ma allora osserviamo che
\[
  \left(\bigcap_{i=1}^N A_i \right) \cap (A \cap B) 
 = \left(\bigcap_{i=1}^N A_i\right) \cap A \cap B \neq \emptyset
\]
in quanto $\{A_1,\dots, A_N, A, B\}$ � una sottofamiglia finita di $\M$.

Per il secondo punto vogliamo invece dimostrare che $\M'\defeq \M\cup
\{A\}$ ha la PIF, sapendo che $A\cap B\neq \emptyset$ per ogni $B\in
\M$. Sia dunque $\F_0$ una sottofamiglia di $\M'$ non interamente
contenuta in $M$. Si avr�
\[
  \F_0 = \{A_1,A_2, \dots, A_N, A\}.
\]
Per il punto precedente, iterato $N-1$ volte, si ha che
$B=A_1\cap\dots\cap A_N\in \M$. Dunque
\[
  \bigcap \F_0 = A_1\cap \dots\cap A_N \cap A = A\cap B \neq \emptyset.
\]

\begin{proposition}\label{prop5}
Sia $\F$ una famiglia di sottoinsiemi di $X$ con la PIF.
Sia $f\colon X\to Y$ una funzione. Allora 
\[
  \F' \defeq \{f(A) \colon A\in \F\}
\]
� una famiglia di sottoinsiemi di $Y$ con la PIF. 
\end{proposition}
\begin{proof}
\`E sufficiente ricordare che vale
\[
  f(A_1) \cap \dots \cap f(A_N) \supset
  f(A_1 \cap A_2 \cap \dots \cap A_N). \qedhere
\]
\end{proof}

\begin{theorem}[Tychonov]
Siano $X_i$ spazi topologici compatti e $T_2$ per ogni $i$ in una generica famiglia di indici $I$.
Allora lo spazio prodotto
\[
  X = \prod_{i\in I} X_i
\]
munito della topologia meno fine che rende continue le proiezioni
\begin{align*}
  \pi_i \colon X &\to X_i \\
  x  &\mapsto x_i
\end{align*}
� compatto e $T_2$.
\end{theorem}
\begin{proof}
Notiamo innanzitutto che $X$ � $T_2$. Siano $x,y \in X$, $x\neq y$.
Allora esiste $i\in I$ tale che $x_i\neq y_i$. Visto che $X_i$ � $T_2$
possiamo trovare $U_i,V_i$ aperti disgiunti in $X_i$ tali che $x_i\in U_i$,
$y_i \in V_i$. Posto $U=\pi_i^{-1}(U_i)$, $V=\pi_i^{-1}(V_i)$ possiamo
facilmente verificare che $U$ e $V$ sono intorni aperti disgiunti dei
punti $x$ e $y$.

Per dimostrare che $X$ � compatto sar� sufficiente dimostrare che ogni
famiglia di chiusi $\F$ con la propriet� PIF ha intersezione non
vuota. Sia dunque $\F$ una tale famiglia. Innanzitutto estendiamo $\F$
ad una famiglia massimale $\M\supset \F$ con la PIF. Ovviamente sar�
$\bigcap \M \subset \bigcap \F$ dunque sar� sufficiente dimostrare che
la famiglia $\M$ ha intersezione non vuota.

Per ogni $i\in I$ definiamo
\[
  \M_i \defeq\{\overline{\pi_i(A)}\colon A\in \M\}.
\]
Per la proposizione~\ref{prop5} la famiglia $\M_i$ ha la PIF e visto
che $X_i$ � compatto ogni famiglia $\M_i$ ha intersezione non vuota. Dunque
per ogni $i\in I$ possiamo trovare $x_i\in X_i$, $x_i \in \bigcap
\M_i$.
Il nostro scopo sar� dimostrare che il punto
\[
  x \defeq (x_i)_{i\in I}
\]
risulta essere elemento dell'intersezione $\bigcap \M$, che dunque dovr� essere non vuota.

Per fare ci� dimostreremo che nella famiglia $\M$ possiamo trovare un sistema
fondamentale di intorni chiusi di $x$.
Sia $W$ un qualunque intorno di $x$. Per come � definita la topologia
su $X$ e per la proposizione~\ref{prop2} 
sappiamo che esistono $i_1, \dots, i_N \in I$ e $V_{i_k}\subset
X_{i_k}$ tali che $V_{i_k}$ � un intorno chiuso di $x_{i_k}$ e
\[
  \bigcap_{k=1}^N \pi_{i_k}^{-1}(V_{i_k}) \subset W.
\]
Osserviamo che per ogni $A\in \M$ si ha $x_{i_k}\in \pi_{i_k}(A)$ e
dunque $A\cap \pi_{i_k}^{-1}(V_{i_k}) \neq \emptyset$.
Dunque $\pi_{i_k}^{-1} (V_{i_k}) \in \M$ per ogni $k=1,\dots, N$ e di
conseguenza 
\[
  v =\bigcap_{k=1}^N \pi_{i_k}^{-1}(V_{i_k}) \in \M.
\]

Dunque per ogni $A \in \M$ e per ogni $W$ intorno di $x$ abbiamo
verificato che $A\cap W\neq \emptyset$ cio� $x\in \bar A = A$, come
volevamo dimostrare.
\end{proof}

\end{document}