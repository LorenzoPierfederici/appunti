\documentclass[italian,a4paper]{article}
\usepackage{babel}
\usepackage{a4}
\usepackage{latexsym}

\newcommand{\R}{\mathbf{R}}
\newcommand{\CC}{\mathcal{C}}
\renewcommand{\epsilon}{\varepsilon}

%% vari package utili
%\usepackage{graphicx}
%\usepackage{pstricks}

%\usepackage{showkeys}
%\PSTricksOff
%\usepackage{pst-node}
%\usepackage{amsmath}
%\usepackage{amsfonts}
%\usepackage{amssymb}
%\usepackage{amsthm}


\newcommand{\hide}[1]{}  %per commentare intere porzioni di testo


% scommentando le seguenti righe i riferimenti ai ``theorem''
% contengono gia' la parola ``Theorem''...
%\makeatletter
%\def\@thm#1#2{%
% \refstepcounter{#1}\@thmlabel{#2}{\csname the#1\endcsname}%
% \@ifnextchar[{\@ythm{#1}{#2}}{\@xthm{#1}{#2}}}\def\@thmlabel#1#2{%
% \def\@currentlabel{#1~#2}}
%\makeatother

\newtheorem{theorem}{Teorema}[section]
\newtheorem{proposition}[theorem]{Proposizione}
\newtheorem{lemma}[theorem]{Lemma}
\newtheorem{definition}[theorem]{Definizione}
\def\profname{Dimostrazione}

\newenvironment{proof}
        {%
%       \begin{list}%
                {}%
                {%
                %\setlength{\rightmargin}{\leftmargin}%
                }%
%       \item%
        \noindent% 
        {\it \profname:}\\%
        }%
        {%
        \hfill $\Box$%
%       \end{list}%
        }

\def\bea{\begin{eqnarray*}}
\def\eea{\end{eqnarray*}}

\title{Studio qualitativo}
\author{Emanuele Paolini}
%%title{Studio qualitativo}
\date{2 luglio 2002}
\begin{document}
\maketitle
Non sempre \`e possibile determinare esplicitamente le soluzione
di una equazione differenziale. Ci proponiamo quindi di trovare dei metodi per
determinare alcune propriet\`a fondamentali della soluzione, senza
doverla trovare esplicitamente. 

\section{Posizione del problema}

Sia $\Omega\subset \R^2$ e sia data $f: \Omega\to\R$ ($\Omega$ \`e
dunque il dominio di $f$). 
Sia $(x_0,y_0)\in \Omega$. 

Consideriamo dunque il problema di Cauchy
\begin{equation}\label{cauchy}
        \left\{\begin{array}{l}
                y'(x) = f(x,y(x))\\
                y(x_0) = y_0
                \end{array}\right.
\end{equation}
e l'equazione differenziale ad esso associata
\begin{equation}\label{eqdiff}
	y'(x)= f(x,y(x)).
\end{equation}

Diremo che una funzione $y:I\to\R$ definita su un intervallo
$I\subset \R$ \`e una \emph{soluzione dell'equazione differenziale}
(\ref{eqdiff}) se 
\begin{enumerate}
\item l'intervallo $I$ contiene pi\`u di un punto (altrimenti il
  problema risulta banale);
\item $y$ \`e derivabile (in quanto nell'equazione differenziale
  compaiono le derivate);
\item per ogni $x\in I$ si ha $(x,y(x))\in \Omega$ (altrimenti
  non posso calcolare $f$);
\item  $y(x)$ soddisfa l'equazione differenziale (\ref{eqdiff}).
\end{enumerate}

Diremo che una soluzione $y$ di (\ref{eqdiff}) definita su un intervallo $I$
\emph{pu\`o essere estesa} ad un intervallo $J$ contenente
strettamente $I$ se su $J$ esiste una soluzione $z$ di (\ref{eqdiff}) che
coincide su $I$ con $y$.

Diremo quindi che una soluzione $y$ dell'equazione differenziale 
(\ref{eqdiff}) definita su un
intervallo $I$ \`e \emph{massimale} se non pu\`o essere in nessun modo
estesa.

Nel caso in cui $x_0\in I$ e $y(x)$ \`e una soluzione di
(\ref{eqdiff}) tale che $y(x_0)=y_0$ diremo che $y(x)$ \`e una
\emph{soluzione del problema di Cauchy} (\ref{cauchy}).

Data una soluzione di (\ref{eqdiff}) definita su un intervallo $I$, se
tale soluzione non \`e
massimale posso estenderla ad un intervallo pi\`u grande
$J$. Iterando il procedimento si trovano soluzioni definite su
intervalli sempre pi\`u grandi finch\`e non si trova
una soluzione che non pu\`o essere ulteriormente estesa. (una
dimostrazioen precisa di questo fatto, se non sappiamo a priori l'unicit\`a della soluzione, richiede il \emph{lemma di Zorn}).

\begin{theorem}
Se esiste una soluzione di (\ref{eqdiff}) allora esiste una soluzione
massimale.
\end{theorem}



\section{Esistenza e unicit\`a delle soluzioni}

Una prima propriet\`a fondamentale per poter poi proseguire nello
studio delle soluzioni \`e determinare se la soluzione esiste e, nel
caso, se \`e unica.

\begin{theorem}[Peano]
Se $f$ \`e continua in un intorno del punto $(x_0,y_0)\in\Omega$
allora esiste una soluzione (locale) di (\ref{cauchy}).
\end{theorem}

Notiamo che se
$f$ \`e continua allora $y'(x)=f(x,y(x))$ \`e anch'essa continua e
dunque le soluzioni $y(x)$ sono funzioni di classe $\CC^1$.

Nei punti in cui $f$ non \`e continua non abbiamo gli strumenti per
dire alcunch\'e. Supporremo quindi nel seguito che $f$ sia continua su
tutto $\Omega$.

\begin{theorem}[Cauchy]\label{unicita}
Se $f$ \'e continua ed \`e lipschitziana rispetto
a $y$ uniformemente rispetto ad $x$ 
in un intorno di $(x_0,y_0)\in\Omega$ allora il problema di
Cauchy~(\ref{cauchy}) ha una unica soluzione massimale.
\end{theorem}

Nel caso in cui la funzione $f$ sia di classe $\CC^1$ il teorema
precedente \`e verificato e quindi \`e garantita l'esistenza e
l'unicit\`a delle soluzioni. 

Di particolare rilevanza \`e il seguente corollario.
\begin{theorem}\label{nontocca}
Siano $y_1(x)$ e $y_2(x)$ due diverse funzioni definite su uno stesso
intervallo $I$ ed entrambi soluzioni dell'equazione differenziale
(\ref{eqdiff}). Supponiamo inoltre che $f$ sia continua e localmente
lipschitziana rispetto ad $y$ uniformemente rispetto ad $x$ (ad
esempio $f\in\CC^1(\Omega)$). Allora $y_1$ e $y_2$ non si toccano mai,
cio\`e per ogni $x\in I$ si ha $y_1(x)\neq y_2(x)$. 
\end{theorem} 
\begin{proof}
Sia $J=\{x\in I: y_1(x)=y_2(x)\}$. Siccome $y_1$ e $y_2$ sono funzioni
continue $J$ risulta essere un sottoinsieme chiuso di $I$. D'altra
parte preso un punto $x_0\in J$, e scelto $y_0=y_1(x_0)=y_2(x_0)$ per
il Teorema~\ref{unicita} esiste un intorno di $x_0$ su cui il problema
(\ref{cauchy}) ha una unica soluzione. Siccome $y_0$ e $y_1$ sono
soluzioni se ne ricava che in tale intorno $y_0$ e $y_1$
coincidono. Dunque tutto un intorno di $x_0$ sta in $J$. Ne consegue
che $J$ \`e anche aperto. Dunque essendo $I$ connesso e sapendo che
$J\neq I$ (in quanto $y_1$ e $y_2$ per ipotesi sono diverse) se ne
deduce che $J=\emptyset$.
\end{proof}

\`E importante notare che nel caso ci sia l'esistenza locale di
soluzioni ($f$ continua), ogni soluzione pu\`o essere estesa fino a
``raggiungere'' il bordo di $\Omega$. Questo \`e precisato nel
seguente teorema che pu\`o essere espresso in questo modo: se c'\`e
esistenza locale delle soluzioni allora la soluzione massimale esce da
ogni compatto assegnato.

\begin{theorem}\label{esce}
Sia $K\subset\Omega$ un compatto e sia $f$ continua su tutto $\Omega$.
Sia $y:I\to\R$ una soluzione massimale di (\ref{cauchy}) con dato iniziale
$(x_0,y_0)\in K$. 
Allora esistono $x_1\ge x_0$ e $x_2<x_0$, $x_1,x_2\in I$ tali
che $(x_1,y(x_1)),(x_2,y(x_2))\in\partial K$.
\end{theorem}
\begin{proof}
Sia $J=\{x\in I : x\ge x_0\}$.
Posto $x_1=\sup J$ si hanno due
possibilit\`a: o $J=[x_0,x_1]$ oppure $J=\left[x_0,x_1\right[$.
Supponiamo per assurdo che per ogni $x\in J$ il punto
$(x,y(x))$ sia interno a $K$. 

Se $J=[x_0,x_1]$ allora, essendo $(x_1,y(x_1))$ un punto interno a
$K$ ed essendo $f$ continua in un intorno di $(x_1,y(x_1))$, 
il teorema di esistenza locale ci
permette di concludere che la soluzione $y$ pu\`o essere estesa ad un
intorno di $x_1$. Questo \`e assurdo perch\'e l'intervallo $I$ era
supposto essere massimale.

Supponiamo invece che $J=\left[x_0,x_1\right[$. Sicuramente
$x_1<\infty$ altrimenti $K$ non sarebbe limitato. Vogliamo dimostrare
che esiste il limite $\displaystyle \lim_{x\to x_1^-} y(x)$. Siano $\displaystyle M=\limsup_{x\to
x_1^-} y(x)$ e $\displaystyle m=\liminf_{x\to x_1^-} y(x)$ e supponiamo per assurdo
che $M>m$. Essendo $y(x)$ continua in $J$ essa assume frequentemente
(per $x\to x_1$) tutti i valori compresi tra $m$ e $M$. In particolare
preso comunque $\epsilon>0$ \`e possibile trovare due punti $x'$ e
$x''$ con $\vert x'-x'' \vert <\epsilon$ tali che $\vert y(x') -
y(x'')\vert > (M-m)/2$. Per il teorema di Lagrange siamo quindi in
grado di trovare una successione $\xi_k\to x_1$ tale che 
$\vert y'(\xi_k)\vert > (M-m)/(2\epsilon)$. 
Siccome $y$ \`e soluzione di (\ref{eqdiff}) si ha dunque 
$\vert f(\xi_k,y(\xi_k))\vert \to\infty$ il che \`e assurdo in quanto la
successione di punti $(\xi_k,y(\xi_k))$ \`e interamente contenuta nel
compatto $K$ ed essendo continua $f$ \`e limitata su $K$.
Dunque deve essere $M=m$ cio\`e $m=\lim_{x\to x_1^-}y(x)$. Essendo $K$
chiuso si ha $(x_1,m)\in K$. Estendiamo quindi $y$ sull'intervallo
chiuso $[x_0,x_1]$ ponendo $y(x_1)=m$. La funzione risultante risulta
essere continua. Inoltre per Lagrange si ha
$(y(x_1)-y(x_1-h))/h = y'(\xi_h) = f(\xi_h,y(\xi_h))\to f(x_1,m)$ e
dunque $y$ \`e derivabile (derivata sinistra) nel punto $x_1$ e vale
$y'(x_1)=f(x_1,y(x_1))$. Dunque effettivamente abbiamo costruito una
estensione della soluzione, e questo contraddice l'ipotesi che $y$
fosse una soluzione massimale.
\end{proof}

\medskip\noindent
\emph{Esempi.}
\begin{enumerate}
\item
Sia $\Omega=\R^2$, $f(x,y)=0$ se $x\neq 0$, $f(x,y)=1$ se $x=0$,
$x_0=0$, $y_0=17$. Allora
il problema di
Cauchy (\ref{cauchy}) non ha alcuna soluzione. Infatti
una eventuale soluzione $y(x)$ dovrebbe essere costante per $x>0$ e
anche per $x<0$. Essendo derivabile $y(x)$ dovrebbe essere continua e
quindi necessariamente dovrebbe essere costante su tutto l'intervallo $I$ di
definizione.  Ma allora si avrebbe $y'(0)=0 \neq f(0,y(0))=1$.

\item
Sia $f(x,y)=0$ se $x<0$, $f(x,y)=1$ se $x\ge 0$.
Allora la funzione
$y(x)=17+x$ \`e una soluzione massimale di (\ref{eqdiff})
sull'intervallo $I=[0,\infty[$. Supponiamo infatti che esista una
soluzione $z(x)$ definita su un intervallo $J=[-\epsilon,\infty[$ e
coincidente con $y$ su $I$. Allora per $x<0$ l'estensione $z$ \`e
costante (in quanto $z'=0$) ed essendo $z$ una funzione continua
dovr\`a essere $z=17$ per $x<0$. Dunque $z$ non \`e derivabile in $0$
e non \`e quindi una soluzione di (\ref{eqdiff}).

\item
Si consideri la funzione $f(x,y)=2\sqrt{\vert y\vert}$ e poniamo
$x_0=0$, $y_0=0$. Si verifica allora
facilmente che la funzione $y(x)=0$ \`e soluzione di
(\ref{cauchy}). D'altra parte anche la funzione definita su tutto
$\R$: $y(x)=x^2$ per $x\ge
0$ e $y(x)=-x^2$ per $x<0$ \`e soluzione di (\ref{cauchy}). Dunque in
questo caso la soluzione non \`e unica, e infatti la funzione $f$ pur
essendo continua non risulta essere lipschitziana rispetto $y$ 
in nessun intorno del punto $y_0$.

\item
Dimostrare che il seguente problema di Cauchy
\[
	\left\{ \begin{array}{l}
		y'(x) = (y^2(x)-1) x\sin x \\
		y(0) = 0
	\end{array}\right.
\]
ammette una soluzione unica $y(x)$ definita globalmente su tutto $\R$.

In questo caso si ha $f(x,y)=(y^2-1)x\sin x$, che \`e una funzione
$\CC^1$ su tutto $\Omega=\R^2$. Vale dunque il teorema di esistenza ed
unicit\`a delle soluzioni. Notiamo inoltre che $y_0(x)=-1$ e $y_1(x)=1$ sono
soluzioni dell'equazione differenziale (\ref{eqdiff}). 
Consideriamo ora il compatto
$K_M=[-M,M] \times [-1,1]$. Per Teorema~\ref{esce} la soluzione
massimale $y(x)$ del problema di Cauchy preso in considerazione deve
uscire da ogni compatto $K_M$. La soluzione $y(x)$ per\`o, per il Teorema~\ref{nontocca} non pu\`o toccare
le due altre soluzioni $y_0(x)$ e $y_1(x)$. Dunque necessariamente
deve toccare $\partial K$ nei due segmenti verticali $x=M$ e
$x=-M$. Questo implica che $M,-M\in I$. Siccome questo \`e vero per
ogni $M>0$ otteniamo che $I=\R$.
\end{enumerate}

\section{Monotonia e punti critici delle soluzioni}

Dopo aver determinato le zone di $\Omega$ su cui vale il teorema di
esistenza e unicit\`a, \`e utile studiare il segno di $f$. Infatti 
le soluzioni $y(x)$ dell'equazione differenziale (\ref{eqdiff})
saranno strettamente crescenti dove $f>0$, strettamente decrescenti
dove $f<0$ e avranno un punto critico dove $f=0$.

Di particolare rilevanza \`e l'insieme $\{f=0\}$ che quando
$f\in\CC^1$ risulta essere, (nei punti in cui $\nabla f \neq 0$)
il grafico di una funzione differenziabile.

Un primo caso notevole \`e il caso in cui $f$ si annulla su una 
retta orizzontale $y=c$. In questo caso la funzione costante $y=c$ \`e una
soluzione dell'equazione differenziale (\ref{eqdiff}). 
In particolare
(nell'ipotesi $f\in\CC^1$) tale costante non pu\`o essere
attraversata dalle altre soluzioni dell'equazione differenziale.

\`E importante capire se le soluzioni dell'equazione differenziali
attraversano o no la curva $\{f=0\}$. Infatti questo ci permette di
determinare il segno della derivata $y'(x)$ e quindi di ottenere
importanti informazioni sull'andamento della soluzione $y(x)$
(monotonia, massimi e minimi relativi...).

Pi\`u in generale \`e interessante capire quando una soluzione di una
equazione differenziale pu\`o attraversare una determinata curva.

\begin{theorem}\label{nonpassa}
Sia $y(x)$ una soluzione dell'equazione differenziale $(\ref{eqdiff})$
definita su un intervallo $I$ e sia $y_0(x)$ una funzione qualunque
definita su $I$, di classe $\CC^1$ e tale che il suo grafico sia
interamente contenuto nel dominio $\Omega$ di definizione di $f$.
Supponiamo che in un punto fissato $x_0$ interno ad $I$ si abbia
$y(x_0)<y_0(x_0)$ e supponiamo inoltre che per ogni $x\in I$ si abbia
$y_0'(x)>f(x,y_0(x))$. Allora per ogni $x>x_0$ si ha $y(x)<y_0(x)$.
\end{theorem}

\begin{proof}
Supponiamo per assurdo che l'insieme $J=\{x\in I: x\ge x_0\
\mathrm{e}\ y(x)=y_0(x)\}$ non sia vuoto. Tale insieme \`e chiuso ed
inferiormente limitato, quindi se non \`e vuoto, ammette minimo. Sia
$\bar x$ il minimo di $J$. Sicuramente $\bar x>x_0$ in quanto $y(\bar
x)=y_0(\bar x)$. Inoltre per ogni $x\in \left[x_0,\bar x\right[$ si ha
$y(x)-y_0(x)<0$ e quindi $y'(\bar x) \ge y_0'(\bar x)$ (in quanto 
il rapporto incrementale sinistro di $y-y_0$ \`e sempre positivo).

Questo \`e assurdo in quanto per ipotesi si ha invece $y'(\bar x) =
f(\bar x, y(\bar x)) = f(\bar x ,y_0(\bar x)) < y_0'(\bar x)$.
\end{proof}

Se ad esempio $y_0(x)$ \`e una funzione 
il cui grafico \`e contenuto in $\{f=0\}$
(cio\`e $f(x,y_0(x))=0$) allora una soluzione $y(x)$ dell'equazione
differenziale pu\`o attraversare la curva $y_0(x)$ dall'alto verso il
basso nei punti in cui $y'_0(x)>0$ e dal basso verso l'alto nei punti
in cui $y'_0(x)<0$.



% \begin{theorem}[orizzontali]
% Sia $y(x)$ una soluzione dell'equazione differenziale (\ref{eqdiff})
% definita su un intervallo $I=\left[x_0,\infty\right[$. Supponiamo
% inoltre che $y'(x)\ge 0$ per ogni $x\in I$.
% \end{theorem}


\medskip\noindent
\emph{Esempi.}

\begin{enumerate}
\item
Mostrare che il seguente problema di Cauchy
\[
\left\{\begin{array}{l}
  y'=1-x y^3\\
  y(0)=0
	\end{array}\right.
\]
ammette una soluzione $y(x)$ definita su tutto $\R$.


Posto $f(x,y)=xy^3$ abbiamo che $f$ si annulla sul grafico della
funzione $y_0(x)=1/\sqrt[3] x$. Per $x>0$ la soluzione \`e crescente
quando si trova al di sotto della funzione $y_0$ ed \`e decrescente
quando si trova al di sopra. Inoltre essendo $y_0'<0$ le soluzioni possono
attraversare la funzione $y_0(x)$ solo passando da sotto a
sopra. 

Sia $y(x)$ la soluzione del problema di Cachy in questione, definita su
un intervallo massimale $I$. Il dato
iniziale \`e $(0,0)\in \{f>0\}$. Dunque la soluzione risulta essere
strettamente crescente in un intorno di $0$. 

Vogliamo dimostrare innanzitutto che necessariamente la soluzione 
incontra entrambi i rami del grafico di $y_0(x)$. Sia infatti
$\epsilon>0$ sufficientemente piccolo da appartenere all'intervallo
$I$ di
esistenza della soluzione e sia $\delta=y(\epsilon)$. 
Consideriamo il compatto $K=\{(x,y):
\epsilon/2 \le x \le 1/\delta^2,\ 0\le y\le 1/sqrt{x}\}$. Il punto
$(\epsilon,y(\epsilon))$ \`e interno al compatto $K$, e il
Teorema~\ref{esce} ci garantisce che la soluzione esce dal compatto
per un qualche $x>\epsilon$. La soluzione per\`o non pu\`o toccare la
retta $y=0$ in quanto all'interno del compatto rimane sempre positiva
e crescente. Non pu\`o neanche toccare il segmento verticale
$x=1/\delta^2$, $y\in[0,\delta]$ in quanto $y(\epsilon)=\delta$ e per
$x>0$ la soluzione \`e strettamente crescente. Dunque necessariamente
la soluzione deve incontrare il grafico della funzione
$y_0(x)=1/\sqrt{x}$ in un punto $\bar x>0$.

Nel punto $x=\bar x$ la soluzione $y(x)$ attraversa la curva
$y_0(x)$. Infatti in tale punto $y'(\bar x)=0$ mentre $y_0'(\bar
x)<0$. Dunque in un intorno destro di $\bar x$ la soluzione si trova
al di sopra della curva $y_0(x)$ e quindi risulta essere decrescente
(in $\bar x$ la soluzione presenta un massimo relativo).

Il Teorema~\ref{nonpassa} ci garantisce inoltre che la soluzione non
pu\`o pi\`u riattraversare il grafico della funzione $y_0$ in quanto
$y'_0<0$. Dunque la soluzione \`e decrescente e limitata. Applicando,
al solito, il Teorema~\ref{esce} si pu\`o quindi mostrare che la
soluzione massimale \`e definita per ogni $x>0$.

Un ragionamento analogo si fa per $x<0$ ottenendo l'esistenza globale,
su tutto $\R$ della soluzione.


\item
Dimostriamo che il problema di Cauchy
\[
\left\{\begin{array}{l}
	y' = -x(y^3 - \sin x) \\
	y(0) = 0
	\end{array}\right.
\]
ammette una soluzione $y(x)$ definita su tutto $\R$. 

Sia $y(x)$ la soluzione del problema di Cauchy e sia $I$ l'intervallo
massimale di definizione. 

Mostriamo innanzitutto che $\vert y(x) \vert < 2$ per ogni $x\in
I$. Infatti per $x>0$ sulla curva $y_1(x)=2$ si ha $f(x,2) = -x
(8-\sin x) < -x < 0 = y_1'(x)$. Dunque per il Teorema~\ref{nonpassa} la
soluzione per $x>0$ non pu\`o mai attraversare la retta
$y=2$. Discorso analogo si fa per la retta $y=-2$ e anche per $x<0$
mostrando che il grafico della  soluzione non pu\`o mai toccare le
rette $y=2$ e $y=-2$ n\'e per $x>0$ n\'e per $x<0$ 
e che quindi $y$ risulta limitata (in realt\`a si
pu\`o essere pi\`u precisi e mostrare che $\vert y\vert \le 1$).

A questo punto si applica, come al solito, il Teorema~\ref{esce} ai
compatti $K_M=[-M,M]\times[-2,2]$ mostrando che la soluzione deve
avere esistenza globale.
\end{enumerate}

\section{Confronto}

\begin{theorem}
Sia $I\subset \R$ un intervallo su cui sono definite due funzioni derivabili
$f(x)$ e $g(x)$. 
Siano $x_0<x_1$ due punti di $I$. Se $f(x_1)<g(x_1)$ e $f(x_2)>g(x_2)$
allora esiste un punto $\bar x \in (x_1,x_2)$ tale che $f(\bar
x)=g(\bar x)$ e $f'(\bar x)\ge g'(\bar x)$.
\end{theorem}

\begin{theorem}
Consideriamo due funzioni $y(x)$ e $z(x)$ soluzioni dei rispettivi
problemi di Cauchy
\[
\begin{cases}
  y'(x)  = f(x,y(x))\\
  y(x_0) = y_0
\end{cases}
\qquad
\begin{cases}
  z'(x) = g(x,z(x))\\
  z(x_0) = y_0
\end{cases}
\] 
su un intervallo $I$ che contiene il punto $x_0$. 
Sia $\Omega\subset \R^2$ un sottoinsieme degli insiemi di definizione
di $f$ e $g$ tale che i grafici delle soluzioni $y(x)$ e $z(x)$ siano
contenuti in $\Omega$ al variare di $x\in I$. 
Supponiamo che per ogni $(x,y)\in\Omega$ si abbia $f(x,y)<g(x,y)$.
Allora per ogni $x\in I$, $x> x_0$ si ha $y(x)< z(x)$ mentre per ogni
$x\in I$, $x<x_0$ si ha $y(x)>z(x)$.
\end{theorem}

\begin{theorem}
Sia $f\colon (x_0,+\infty)\to \R$ una funzione di classe $\mathcal C^1$ tale
che il limite
\[
  \lim_{x\to +\infty} f'(x) = \ell 
\]
esiste, finito o infinito. Se $\ell\neq 0$ allora $f$ non pu\`o avere
un asintoto orizzontale per $x\to +\infty$.
\end{theorem}

\end{document}
