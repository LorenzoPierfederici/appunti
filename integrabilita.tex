\documentclass[italian,a4paper]{article}
\usepackage{babel}
\usepackage{a4}
\usepackage{amssymb}
\usepackage{latexsym}
\newcommand{\R}{\mathbb{R}}
\renewcommand{\epsilon}{\varepsilon}
%%title{Integrabilit\`a di funzioni non continue}
\title{\bf Integrabilit\`a di funzioni non continue}
\author{E. Paolini}
\date{29 marzo 2003}

\newtheorem{theorem}{Teorema}
\newtheorem{proposition}[theorem]{Proposizione}
\newtheorem{example}[theorem]{Esempio}
\newtheorem{corollary}[theorem]{Corollario}
\newtheorem{lemma}[theorem]{Lemma}
\newtheorem{definition}[theorem]{Definizione}
\newtheorem{exercise}[theorem]{Esercizio}
\newenvironment{proof}
        {%
%       \begin{list}%
                {}%
                {%
                %\setlength{\rightmargin}{\leftmargin}%
                }%
%       \item%
        \noindent% 
        {\it Dimostrazione:}\\%
        }%
        {%
        \hfill$\square$%
%       \end{list}%
        }


\begin{document}
\maketitle

In questo documento studieremo l'integrabilit\`a secondo Riemann di
funzioni non continue. Riprendiamo brevemente la definizione di
integrale di Riemann (d'ora in avanti l'integrabilit\`a sar\`a sempre
intesa secondo Riemann).

Consideriamo una funzione limitata $f\colon [a,b]\to \R$. 
Una \emph{partizione} di $[a,b]$ \`e una sequenza finita e crescente di punti
$P=(x_0,x_1,\ldots,x_N)\subset [a,b]$ con $x_0=a$ e $x_N=b$.
Fissata una partizione $P$ e posto $I_k=[x_k,x_{k+1}]$ 
definiamo quindi le somme superiori e
inferiori:
\[
	S_P = \sum_{k=0}^{N-1} (x_{k+1}-x_k) \sup_{x\in I_k} f(x),\quad
	s_P = \sum_{k=0}^{N-1} (x_{k+1}-x_k) \inf_{x\in I_k} f(x),\quad.
\]
Definiamo poi $S=\inf_P S_P$ e $s=\sup_P s_P$ dove $P$ varia tra tutte
le possibili partizioni di $[a,b]$.
Essendo $S_P\ge s_P$ per ogni partizione $P$ si ha chiaramente $S\ge
s$.
Nel caso in cui si abbia $S=s$ diremo che $f$ \`e integrabile su
$[a,b]$ e indicheremo con
\[
	\int_a^b f(x)\, dx = S = s
\]
l'integrale di $f$ su $[a,b]$.

Notiamo che la definizione appena vista di integrabilit\`a pu\`o
essere applicata solamente a funzioni limitate definite su un
intervallo chiuso e limitato. Pu\`o risultare utile introdurre anche
il concetto di \emph{locale integrabilit\`a} che si applica ad una
classe pi\`u ampia di funzioni.
 
\begin{definition}[locale integrabilit\`a]
Sia $I$ un intervallo di $\R$ e sia $f\colon I \to \R$. Diremo che $f$
\`e localmente integrabile se per ogni intervallo chiuso e limitato
$[a,b]\subset I$ la funzione $f$ ristretta all'intervallo $[a,b]$
\`e limitata e integrabile.
\end{definition}

\begin{definition}[funzione integrale]
Sia $I$ un intervallo di $\R$ e sia $f\colon I \to \R$ una funzione
localmente integrabile.
Scelto un punto $x_0\in I$ possiamo quindi definire la \emph{funzione
integrale} $F\colon I\to\R$
\[
	F(x)=\int_{x_0}^x f(t)\, dt.
\]
\end{definition}

\begin{definition}[primitiva]
Sia $I$ un intervallo di $\R$ e sia $f\colon I\to \R$. Una funzione
$F\colon I\to \R$ si dice \emph{primitiva} di $f$ se vale
\[
	F'(x) = f(x)\qquad \forall x\in I.
\]
\end{definition}

Il seguente teorema mette in corrispondenza le primitive con le
funzioni integrali.

\begin{theorem}[Teorema fondamentale del calcolo integrale]
Sia $I$ un intervallo di $\R$ e sia $f\colon I\to \R$ una funzione
continua. Allora 
\begin{enumerate}
\item
$f$ \`e localmente integrabile;
\item
ogni funzione integrale \`e una primitiva di $f$;
\item
due diverse primitive di $f$ differiscono per una costante;
\item
se $F$ \`e una qualunque primitiva di $f$ si ha
\[
	\int_a^b f(x)\, dt = F(b) - F(a).
\]
\end{enumerate}
\end{theorem}

\begin{theorem}
Sia $f\colon [a,b]\to \R$ una funzione limitata e integrabile. 
\begin{enumerate}
\item
	Ogni funzione integrale di $f$ \`e continua, anzi \`e Lipschitziana;
\item	
	due diverse funzioni integrali di $f$ differiscono per una costante.
\end{theorem}

\begin{theorem}
Sia $f\colon [a,b]\to\R$ una funzione limitata.
Se $f$ ristretta all'intervallo aperto $\left]a,b\right[$ \`e
localmente integrabile allora $f$ \`e integrabile
sull'intero intervallo $[a,b]$. Inoltre si ha 
\[
	\int_a^b f(x)\, dx =\lim_{\epsilon\to
	0}\int_{a+\epsilon}^{b-\epsilon} f(x)\, dx.
\]
\end{theorem}

\begin{theorem}
Sia $f\colon [a,b]\to\R$ una funzione limitata con un numero finito di
discontinuit\`a. Allora $f$ \`e integrabile. Inoltre se $F$ \`e una
funzione integrale di $f$ si ha $F'(x)=f(x)$ 
in ogni punto $x$ in cui $f$ \`e continua.
\end{theorem}

\end{document}
