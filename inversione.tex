\documentclass[italian]{article}

\newcommand{\tthdump}[1]{#1}                                                   
\tthdump{\newcommand{\tth}[1]{}}
%%tth:\newcommand{\tth}[1]{#1}

\usepackage{babel}
\usepackage{a4}
\usepackage{latexsym}
\usepackage{amsmath,amssymb,amsthm}

\usepackage[utf8]{inputenc}

\newcommand{\hide}[1]{}  %per commentare intere porzioni di testo

\newtheorem{theorem}{Teorema}
%[section]
\newtheorem{proposition}[theorem]{Proposizione}
\newtheorem{corollary}[theorem]{Corollario}
\newtheorem{lemma}[theorem]{Lemma}
\newtheorem{definition}[theorem]{Definizione}
\theoremstyle{remark}
\newtheorem{example}[theorem]{Esempio}
\newtheorem{exercise}[theorem]{Esercizio}


%%tth: \def\mathbb\mathbf
\def\text#1{{\mbox{\rm #1}}}            %% testo nelle formule
\newcommand{\e}{{\mathrm e}}            %% costante di nepero
\newcommand{\Id}{{\mathrm{id}}}          %% identita
\newcommand{\R}{{\mathbb R}}            %% numeri reali
\newcommand{\CC}{{\mathcal C}}           %% funzioni continue
\newcommand{\M}{{\mathbb M}}         %% matrici
\newcommand{\N}{{\mathbb N}}           %% numeri interi
\newcommand{\C}{{\mathbb C}}           %% numeri complessic
\newcommand{\B}{{\mathbb B}}  %% palla
\newcommand{\D}{{\mathbb D}}  %% disco
\renewcommand{\S}{{\mathbb S}}  %% sfera
\renewcommand{\H}{{\mathcal H}}         %% misura di Hausdorff
\newcommand{\F}{{\mathcal F}}           %% frontiera ridotta
\newcommand{\Z}{{\mathbf Z}}            %% interi
\renewcommand{\l}{\ell}           %% spazi l - piccolo
\renewcommand{\epsilon}{{\varepsilon}}  %% epsilon
\renewcommand{\phi}{{\varphi}}          %% phi
%\newcommand{\Subset}{\subset\subset}    %% contenuto-compatto
%\renewcommand{\subset}{{\subseteq}}     %% sottoinsieme (largo)
%\renewcommand{\supset}{{\supseteq}}     %% soprainsieme (largo)
\newcommand{\nsubset}{{\not\!\subseteq}}%% non sottoinsieme
\renewcommand{\d}{{\text{d}}}         %% differenziale negli integrali
\newcommand{\diffsim}{\triangle}        %% differenza simmetrica
\newcommand{\loc}{\mathrm{loc}}         %% 'loc'
\newcommand{\defeq}{:=}                 %% uguale per definizion
\newcommand{\res}               {\mathop{\hbox{
                                \vrule height 7pt width .5pt depth 0pt
                                \vrule height .5pt width 6pt depth 0pt}}
                                \nolimits}
\newcommand{\tr}{\mathrm{tr}}
\newcommand{\eps}{\epsilon}
\def\bea{\begin{eqnarray*}}
\def\eea{\end{eqnarray*}}

\newcommand{\open}[1]{\stackrel{\circ}{#1}}

%%%%%%%%%%%%%%%%%%%%%%%%%%%%%%%%%%%%%%%%%%%%%%%%%%%%%%%%%%%%
\title{Il teorema di invertibilit\`a locale e il teorema del Dini per
  i sistemi}
\author{E. Paolini}
%%%%%%%%%%%%%%%%%%%%%%%%%%%%%%%%%%%%%%%%%%%%%%%%%%%%%%%%%%%%

%%%%%%%%%%%%%%%%%%%%%%%%%%%%%%%%%%%%%%%%%%%%%%%%%%%%%%%%%%%%
\begin{document}
%%%%%%%%%%%%%%%%%%%%%%%%%%%%%%%%%%%%%%%%%%%%%%%%%%%%%%%%%%%%
%%%%%%%%%%%%%%%%%%%%%%%%%%%%%%%%%%%%%%%%%%%%%%%%%%%%%%%%%%%%
\maketitle
%%%%%%%%%%%%%%%%%%%%%%%%%%%%%%%%%%%%%%%%%%%%%%%%%%%%%%%%%%%%
\begin{theorem}[disuguaglianza di Cauchy-Schwarz vettoriale]
Sia $f\colon [a,b] \to \R^m$ una funzione integrabile. Allora
\[
  \left|\int_a^b f(t)\, dt \right| \le \int_a^b |f(t)|\, dt.
\]
\end{theorem}
\begin{proof}
Poniamo $v=\int_a^b f(t)\, dt$. 
Allora si ha
\begin{align*}
  \left|\int_a^b f(t)\, dt\right|\, |v| 
  & =  \left|\left(\int_a^b f(t)\, dt, v\right)\right|
    = \left|\int_a^b (f(t),v)\, dt\right| \\
  &\le \int_a^b |f(t)|\, |v|\, dt
    = \int_a^b |f(t)|\, dt \, |v|.
\end{align*}
Se fosse $v=0$ il teorema sarebbe ovvio, in caso contrario si pu\`o
semplificare $|v|$ nella precedente catena di disuguaglianze,
ottenendo quindi la tesi.
\end{proof}

\begin{theorem}[disuguaglianza di Cauchy-Schwarz matriciale]
Sia $A\colon [a,b]\to M^{m\times n}$ una funzione integrabile a valore
matrici. Allora 
\[
 \left\Vert \int_a^b A(t)\, dt \right\Vert \le \int_a^b ||A(t)||\, dt.
\]
\end{theorem}
\begin{proof}
Poniamo $\bar A = ||\int_a^b A(t)\, dt||$ e sia $\bar v$ tale che
$||\bar v||\le 1$ e $||\bar A|| = |\bar A \bar v|$. Allora si ha
\begin{align*}
\left\Vert\int_a^b A(t)\, dt\right\Vert 
  &= \left| \int_a^b A(t)\, dt\, \bar v\right|
  =  \left| \int_a^b A(t)\,\bar v \, dt \right|\\
  &\le \int_a^b |A(t)\, \bar v|\, dt
   \le \int_a^b ||A(t)||\, dt.
\end{align*}
\end{proof}


\begin{theorem}[criterio di Lipschitz]
Sia $f\colon \Omega\subset \R^n\to \R^m$ una funzione di classe
$\CC^1$ definita su un insieme convesso $\Omega$ 
e supponiamo che si abbia $L=\sup_{x\in \Omega} \Vert
Df(x)\Vert <+\infty$. Allora la funzione $f$ \`e $L$-Lipschitziana,
ossia
\[  
  | f(x')-f(x)| \le L | x'-x| \qquad \forall x,x'\in \Omega.
\]
\end{theorem}
\begin{proof}
Siano $x,x'\in \Omega$ fissati e
consideriamo la funzione $g(t)=f(tx'+(1-t)x)$. Chiaramente $g\colon
[0,1]\to \R^m$ \`e di classe $\CC^1$ e si ha
\[
  g'(t) = Df(tx'+(1-t)x)\, (x'-x).
\]
Dalla formula fondamentale del calcolo integrale si ottiene
\begin{align*}
  f(x')-f(x) &= g(1)-g(0) = \int_0^1 g'(t)\, dt
  =\int_0^1 Df(tx'+(1-t)x)\, dt\, (x'-x)
\end{align*}
da cui segue
\[
  |f(x')-f(x)| \le \int_0^1 \Vert Df(tx'+(1-t)x)\Vert \, dt\,
    |x'-x|
    \le L |x'-x|
\]
come volevasi dimostrare.
\end{proof}

%%%%%%%%%%%%%%%%%%%%%%%%%%%%%%%%%%%%%%%%%%%%%%%%%%%%%%%%%%%%
\begin{theorem}[inversione locale]
%%%%%%%%%%%%%%%%%%%%%%%%%%%%%%%%%%%%%%%%%%%%%%%%%%%%%%%%%%%%
Sia $\Omega$ un aperto di $\R^n$, $f\in \CC^1(\Omega,\R^n)$,
sia $x_0\in \Omega$ e supponiamo che $Df(x_0)$ 
sia una matrice invertibile.
Allora esiste un intorno aperto $U$ di $x_0$ e un intorno aperto $V$ di $f(x_0)$
per cui $f(U)=V$ e
\[
        f_{|U}\colon U\to V
\]
\`e bigettiva, la sua inversa $f^{-1}\colon V\to U$ \`e differenziabile e se $f(x)=y$ (con $x\in
U$) si ha
\[
        D(f^{-1}(y))=(Df(x))^{-1}.
\]
\end{theorem}
%%%%%%%%%%%%%%%%%%%%%%%%%%%%%%%%%%%%%%%%%%%%%%%%%%%%%%%%%%%%
\begin{proof}
%%%%%%%%%%%%%%%%%%%%%%%%%%%%%%%%%%%%%%%%%%%%%%%%%%%%%%%%%%%%
Consideriamo la matrice $A=(Df(x_0))^{-1}$. 
Siccome $\Omega$ \`e aperto e $Df\colon
\Omega\to \R^n$ \`e continuo, possiamo scegliere $\rho>0$ in modo che
\[    
    \overline{\B_\rho(x_0)}\subset \Omega
\qquad \text{e} \qquad
        \Vert Df(x)-Df(x_0)\Vert \le \frac{1}{2\Vert A\Vert}
                \quad\forall x\in \overline{\B_\rho(x_0)}.
\]
Poniamo $r=\frac{\rho}{2\Vert A\Vert}$ e
$y_0=f(x_0)$.  Fissato $y\in \B_r(y_0)$
consideriamo l'applicazione
\bea
        T\colon \overline{\B_\rho(x_0)}&\to& \R^n\\
        T(x)&=& x + A(y-f(x)).
\eea

Si ha, per $x\in \overline{\B_\rho(x_0)}$, (ricordiamo che $\Id = A\, Df(x_0)$)
\[
        \Vert DT(x)\Vert=\Vert \Id-A\, Df(x)\Vert \le 
        \Vert A \Vert\cdot\Vert Df(x_0)-Df(x)\Vert
        \le \frac{1}{2}
\]
e quindi, per il criterio di Lipschitz, la funzione $T$ \`e $\frac 1
2$-Lipschitziana (ossia una contrazione).

Verifichiamo anche che $T(\overline{\B_\rho(x_0)})\subset \overline{\B_\rho(x_0)}$.
Infatti, per $x\in \overline{\B_\rho(x_0)}$, si ha
        \begin{align*}
        |T(x)-x_0|&\le |T(x)-T(x_0)|+|T(x_0)-x_0|\\
        &\le \frac 1 2 |x-x_0| + |A (y-y_0)|
        \le \frac \rho 2 + \Vert A\Vert  r
        = \rho.
        \end{align*}

Dunque $T\colon\overline{\B_\rho(x_0)}\to \overline{\B_\rho(x_0)}$ \`e una
contrazione, $\overline{\B_\rho(x_0)}$ \`e uno spazio metrico completo
e quindi, per il Teorema delle contrazioni, 
esiste un unica soluzione dell'equazione
        \[
        T(x)=x.
        \]

Dato $y\in \B_r(y_0)$ possiamo dunque trovare un unico 
$x\in \overline{\B_\rho(x_0)}$ che risolve $f(x)=y$, chiamiamo
$g$ l'applicazione $y\mapsto x$
        \[
        g\colon \B_r(y_0) \to \overline{\B_\rho(x_0)}, \qquad f(g(y))=y.
        \]

Consideriamo ora l'insieme $\Sigma = f(\partial B_\rho(x_0))$. Essendo
$f$ continua sappiamo che $\Sigma$ è compatto.
Inoltre $y_0 \not \in \Sigma$ in quanto c'è un unico punto
$x\in \overline{B_\rho(x_0)}$ per cui $f(x)=y_0$ e tale punto è $x_0$.
Dunque $V = B_r(y_0)\setminus \Sigma$ è aperto.
Posto $U = g(V)$
risulta quindi che $f_{|U}\colon U\to V$ \`e
bigettiva e ha $g$ come inversa.
Inoltre per costruzione abbiamo evitato che $U$ possa contenere punti
di $\partial \B_\rho(x_0)$ e dunque risulta $U \subset \B_\rho(x_0)$.

Vogliamo ora mostrare che $U$ è aperto.
Dato un qualunque $x \in U$ poniamo $y= f(x)$. Visto
che $U=g(V)$ sappiamo che $y \in V$. Siccome $V$ è aperto esiste
$\epsilon>0$ tale che $\B_\epsilon(y)\subset V$
ed essendo $f$
continua possiamo trovare $\delta>0$ tale che
$f(\B_\delta(x)) \subset \B_\epsilon(y) \subset V$.
Essendo inoltre $x\in U \subset \B_\rho(x_0)$, pur di rimpicciolire
ulteriormente $\delta$ possiamo supporre anche che
$\B_\delta(x)\subset B_\rho(x_0)$.
Risulta dunque che $f(B_\delta(x))\subset V$ da cui $B_\delta
=g(f(B_\delta)) \subset U$.
Dunque $U$ contiene un'intorno
di ogni suo punto, ed è dunque aperto.


Vogliamo ora dimostrare che $g$ \`e differenziabile in ogni punto
$y\in \B_r(y_0)$ e che il suo differenziale $Dg(y)$ \`e uguale a
$Df(x)^{-1}$ con $x=g(y)$; ossia
\[
\lim_{y'\to y} \frac{g(y')-g(y) - Df(x)^{-1}(y'-y)}{|y'-y|}=0.
\]
Dato $y'\in \B_r(y_0)$, posto $x'=g(y')$ stiamo considerando la
seguente quantit\`a
\begin{align*}
\frac{g(y')-g(y) - Df(x)^{-1}(y'-y)}{|y'-y|} 
=
\frac{x'-x - Df(x)^{-1}(f(x')-f(x))}{|y'-y|}
\end{align*}
che per la differenziabilit\`a di $f$ in $x$ possiamo scrivere come
\begin{equation}
\label{eq_inv_diff}
\begin{aligned}
&=
\frac{x'-x - Df(x)^{-1}(Df(x)(x'-x)+o(|x'-x|))}{|y'-y|}\\
&=
\frac{Df(x)^{-1} o(|x'-x|)}{|y'-y|}
=
\frac{Df(x)^{-1} o(|x'-x|)}{|x'-x|}\cdot \frac{|x'-x|}{|y'-y|}.
\end{aligned}
\end{equation}
Vogliamo ora dimostrare che per $y'\to y$ si ha $x'\to x$ cosicch\'e 
il primo termine di quest'ultimo prodotto tende
a zero per la definizione di $o$ ``piccolo''. Contemporaneamente 
dimostreremo che il secondo termine \`e limitato.

Per fare questo consideriamo la mappa $T$ definita in
precedenza e per la quale si ha $T(x)=x$. 
Ricordando che $T$ \`e $1/2$-lipschitziana si ha
\begin{equation}\label{erro}
\begin{aligned}
\frac 1 2 |x'-x| &\ge |T(x')-T(x)| = |x'+A(y-f(x'))-x|\\
&\ge |x'-x| - |A(y-f(x'))|
= |x'-x| - \Vert A \Vert |y-y'|
\end{aligned}
\end{equation}
da cui
\[
|x'-x| \le 2\Vert A\Vert \,|y'-y|.
\]
Questo da un lato ci dice che se $y'\to y$ allora anche $x'\to x$ e
dall'altro ci dice che il rapporto $|x'-x|/|y'-y|$ \`e limitato.
Dunque la quantit\`a in~\eqref{eq_inv_diff} tende a zero, 
la funzione $g$ \`e differenziabile in $y$ e il suo
differenziale \`e $Dg(y)=Df(x)^{-1}$.

Per concludere che $g\in \CC^1$ rimane solo da provare che il 
differenziale $Dg$ \`e una funzione continua. Ma dato che
$Dg(y)=Df(g(y))^{-1}$, si osserva che essendo $g$ continua (in quanto
differenziabile), essendo $Df$ continua (per ipotesi) ed essendo
continua anche l'operazione di inversione di una matrice, la funzione
$Dg$ deve essere continua.
\end{proof}

%%%%%%%%%%%%%%%%%%%%%%%%%%%%%%%%%%%%%%%%%%%%%%%%%%%%%%%%%%%%
\begin{theorem}[Dini, funzione implicita]
%%%%%%%%%%%%%%%%%%%%%%%%%%%%%%%%%%%%%%%%%%%%%%%%%%%%%%%%%%%%
Sia $\Omega$ un aperto di $\R^n\times \R^m$ e sia $f\in
\CC^1(\Omega,\R^m)$. Sia $(x_0,y_0)\in\Omega\subset \R^n\times \R^m$.
Se la matrice $D_yf(x_0,y_0)$ delle derivate di $f(x,y)$ rispetto alle
variabili $y$
\[
D_yf(x_0,y_0)=\left(\frac{\partial f^j}{\partial y^k} (x_0,y_0)\right)_{j,k}
\qquad j=1,\ldots,m\quad k=1,\ldots,m
\]
\`e invertibile, allora esiste un intorno $U\subset \R^n$ di $x_0$
e una funzione $g\in\CC^1(U,\R^m)$
tale che
\[
        f(x,g(x))=f(x_0,y_0)\qquad \forall x\in U.
\]
Inoltre, per $y=g(x)$ si ha
\[
  Dg(x) = -D_y f(x,y)^{-1} D_x f(x,y).
\]
\end{theorem}
%%%%%%%%%%%%%%%%%%%%%%%%%%%%%%%%%%%%%%%%%%%%%%%%%%%%%%%%%%%%
\begin{proof}
%%%%%%%%%%%%%%%%%%%%%%%%%%%%%%%%%%%%%%%%%%%%%%%%%%%%%%%%%%%%
Consideriamo la funzione $\tilde f\in\CC^1(\Omega,\R^n\times\R^m)$
\[
        \tilde f(x,y)=(x,f(x,y)).
\]
si trova facilmente che 
\[      
        D\tilde f(x,y) =
        \left(
        \begin{array}{c|c}
        \Id & 0 \\ \hline
        D_x f & D_y f
        \end{array}
        \right)
\]
dove $\Id$ \`e la matrice identit\`a $n\times n$. 
Dunque essendo $D_y f(x_0,y_0)$ invertibile,
 anche $D\tilde f(x_0,y_0)$ lo \`e e
posso applicare il teorema di invertibilit\`a locale a $\tilde f$.
Dunque esister\`a una funzione $\tilde g$ di classe $\CC^1$ in un intorno di
$(x_0,f(x_0,y_0)$ 
tale che $\tilde f(\tilde
g(x,y))=(x,y)$.
Posto $\tilde g(x,y)=(g_1(x,y),g_2(x,y))$ con $g_1$ a valori in
$\R^n$, e $g_2$ a valori in $\R^m$ si ha dunque
\[
        (x,y) 
        = \tilde f(g_1(x,y),g_2(x,y))
        = (g_1(x,y),f(g_1(x,y),g_2(x,y)))
\]
da cui si ottiene $g_1(x,y)=x$ e
$f(x,g_2(x,y))=y$.
Baster\`a dunque porre $g(x)=g_2(x,f(x_0,y_0))$ per ottenere $f(x,g(x))=f(x_0,y_0)$.

Per dimostrare la formula di calcolo di $Dg$ \`e sufficiente
calcolare le derivate rispetto a $x$ dell'equazione
\[
  f(x,g(x))=f(x_0,y_0)
\]
per ottenere
\[
  D_x f(x,g(x)) + D_y f(x,g(x)) Dg(x)=0
\]
da cui la tesi.
\end{proof}
\end{document}

\begin{description}
\item[17.3.2006] Ho corretto un errore tipografico nell'enunciato del
  Teorema del Dini: $Dg= - D_y f^{-1} D_x f$.
\item[15.5.2006] Nell'equazione \eqref{erro} c'era erroneamente
  scritto $Df(x)$ invece che $A=Df(x_0)^{-1}$.
\item[29.4.2008] 
  Aggiunti teoremi di tipo Cauchy-Schwarz.
  Sistemata la verifica che $U$ \`e aperto. Mancava una
  tilde nella dimostrazione del teorema del Dini.
\item[31.7.2017]
  Sistemato errore nella dimostrazione che l'insieme $U$ è aperto
  (segnalazione Tortorelli).
\end{description}
