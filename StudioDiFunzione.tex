\documentclass[italian,a4paper]{article}
\usepackage{amsmath,amssymb,amsthm}
\usepackage{babel,a4}
\usepackage[utf8]{inputenc}
\title{Studio di funzione}

\begin{document}
\begin{enumerate}

\item[dominio] se la funzione viene data tramite una espressione, si
  intende che il dominio della funzione è il \emph{campo di esistenza}
  della espressione, ovvero l'insieme degli $x\in \mathbb R$ per cui
  l'espressione $f(x)$ è ben definita.

\item[simmetrie]

\item[limiti]
  In ogni punto di accumulazione del dominio, occorre determinare il
  limite della funzione. Ovviamente se la funzione è continua il
  limite sarà uguale al valore, quindi in tali punti non sarà
  necessario fare un calcolo esplicito.
  \begin{enumerate}
  \item[asintoto verticale] se in un punto $x_0 \in \mathbb R$ si ha
    \[
    \lim_{x\to x_0} f(x) = +\infty,\qquad\text{oppure}\qquad
    \lim_{x\to x_0} f(x) = -\infty
    \]
    diremo che in $x_0$ si ha un \emph{asintoto verticale}. Lo stesso
    vale se il limite risulta da destra o da sinistra.
  \item[asintoto orizzontale] se
    \[
    \lim_{x\to +\infty} f(x) = \ell,\qquad\text{oppure}\qquad
    \lim_{x\to -\infty} f(x) = \ell
    \]
    diremo che la funzione ammette un \emph{asintoto orizzontale}
  \item[asintoto obliquo] se esistono $m\neq 0$ e $q$ tali che
    \[
    \lim_{x\to +\infty} f(x) - (mx+q) = 0,\qquat\text{oppure}\qquad
    \lim_{x\to -\infty} f(x) - (mx+q) = 0
    \]
    diremo che la retta $y=m x + q$ è un \emph{asintoto
      obliquo}.
  \item[estensione per continuità]
    Se la funzione non è definita in $x_0\in \mathbb R$ ma esiste ed è
    finito, il limite
    \[
    \lim_{x\to x_0} f(x) = \ell
    \]
    diremo che la funzione può essere estesa per continuità al punto
    $x_0$.
    Se invece la funzione è definita ma non è continua in $x_0$ e il
    limite di cui sopra esiste ed è finito, diremo che la funzione ha
    una \emph{discontinuità eliminabile} in $x_0$.
  \item[discontinuità a salto]
    Se in un punto $x_0\in \mathbb R$ i limiti destro e sinistro sono
    entrambi finiti ma diversi, diremo che in $x_0$ la funzione ha una
    \emph{discontinuità a salto}.
  \end{enumerate}

  \item[segno] i punti in cui $f$ si annulla vengono chiamati
    \emph{zeri} della funzione.

  \item[punti estremali]
    I punti in cui $f$ assume il valore $\sup f$, vengono chiamati
    \emph{punti di massimo assoluto} di $f$, i punti in cui $f$ assume
    il valore $\inf f$ sono \emph{punti di minimo assoluto} di $f$. Se
    tali punti esistono si dice che la funzione \emph{ammette massimo/minimo}.
    Un punto $x_0$ si dice essere un \emph{massimo relativo (o
      locale)} se esiste un intorno $I$ di $x_0$ per cui $x_0$ risulta
    essere un massimo assoluto della funzione ristretta a tale
    intorno. In maniera analoga si definiscono i punti di \emph{minimo
      relativo (o locale)}.

  \item[monotonia]
    la funzione si dice essere \emph{crescente} su un insieme $I$ se
    la funzione ristretta a tale insieme è crescente (cioè se dati
    $x,y\in I$ si ha $x\le y \Rightarrow f(x)\le f(y)$).

  \item[retta tangente]
    Se in un punto $x_0$ esiste il limite del rapporto incrementale di
    $f$
    \[
    \lim_{x\to x_0} \frac{f(x)-f(x_0)}{x-x_0} = m
    \]
    diremo che il grafico della funzione $f$ ammette \emph{retta
      tangente} nel punto $x_0$. In particolare se $m$ è finito si ha
    $m=f'(x_0)$ e l'equazione della retta tangente è $y-f(x_0) =
    f'(x_0)(x-x_0)$. Se invece $m$ è infinito la retta tangente è
    la retta verticale: $x=x_0$. Nel caso in cui il limite di cui
    sopra esista da destra ($x\to x_0^+$) o da sinistra ($x\to x_0^-$)
    si può parlare di retta tangente \emph{destra} o \emph{sinistra}
    nel punto $x_0$.

  \item[punti angolosi]
    Se in un punto $x_0$ la retta tangente destra e sinistra esistono
    ma sono diverse, diremo che in corrispondenza del punto $x_0$
    c'è un \emph{punto
      angoloso} del grafico di $f$. Se da destra e da sinistra nel
    punto $x_0$ i limiti sono diversi ma entrambi infiniti, diremo che
    nel punto $x_0$ c'è un \emph{punto di cuspide} per il grafico di $f$.

  \item[punto di flesso]
    Un punto $x_0$ in cui la funzione ammette retta tangente si dice
    essere un \emph{punto di flesso} se la retta tangente è verticale
    oppure se la funzione $f(x) - (f(x_0) + f'(x_0)(x-x_0))$ assume
    segno opposto in un intorno destro e in un intorno sinistro di
    $x_0$ (in parole povere: il grafico attraversa la retta tangente).
    
  \item[studio della derivata]
    gli zeri di $f'$ vengono chiamati \emph{punti critici} o
    \emph{punti stazionari} di $f$. Osserviamo che per i criteri di
    monotonia, se la funzione è derivabile in un intorno di un punto
    critico isolato,
    quel punto sarà necessariamente un massimo relativo, un minimo
    relativo oppure un flesso orizzontale. 

\end{enumerate}
\end{document}
