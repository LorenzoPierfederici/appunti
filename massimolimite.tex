\documentclass[italian,a4paper]{article}
\usepackage{babel}
\usepackage{a4}
\newcommand{\R}{\mathbf{R}}
\renewcommand{\epsilon}{\varepsilon}
\title{Brevi note su massimo e minimo limite}
%%title{Brevi note su massimo e minimo limite}
\date{18 aprile 2002}
\begin{document}
\maketitle
Consideriamo per semplicit\`a solo il caso $x\to +\infty$, gli altri
casi si trattano in maniera analoga.


\section*{Definizione}

Sia data una funzione $f: \R\to \R$. Possiamo allora definire
\[
L_f(x)=\sup_{y>x} f(y),\quad l_f(x)=\inf_{y>x} f(y).
\]

Chiaramente si ha $l_f(x)\le L_f(x)$ e si verifica facilmente che
$l_f$ \`e crescente mentre $L_f$ \`e decrescente. Definiamo dunque
\[
\liminf_{x\to +\infty} f(x)= \lim_{x\to\infty} l_f(x) = \inf_x l_f(x),
\]
\[
\limsup_{x\to +\infty} f(x)= \lim_{x\to\infty} L_f(x) = \sup_x L_f(x).
\]

Se la funzione $f$ \`e superiormente limitata si avr\`a $\liminf f\le \limsup
f<+\infty$ se \`e inferiormente limitata si avr\`a 
$\limsup f \ge \liminf f>-\infty$
in generale per\`o potr\`a essere $\limsup f,\liminf f \in [-\infty,+\infty]$.

Posti $I_f(x)=[l_f(x),L_f(x)]$ notiamo che presa una qualunque
successione $x_k\to\infty$ e fissato comunque $x$ grande a piacere si
ha che $f(x_k)\in I_f(x)$ definitivamente in $k$ (cio\`e per $k$
sufficientemente grande). Dunque se $f(x_k)\to l$ comunque si scelga $x$ si ha $l\in
I_f(x)$ e quindi passando al limite $l\in [\liminf f,\limsup f]$.
Dunque ogni sottosuccessione estratta da $f(x)$ converge ad un valore
compreso tra $\liminf$ e $\limsup$. D'altra parte \`e possibile
trovare due particolari sottosuccessioni $X_k\to \infty$,
$x_k\to\infty$ tali che $f(X_k)\to \limsup f$ e $f(x_k)\to \liminf
f$. Infatti dato $k$ intero \`e
sempre possibile trovare un punto $x_k\ge k$ tale che
$f(x_k)>L_f(k)-1/k$ (ricordiamo che $L_f(k)=\sup_{x>k} f(x)$). Si ha
dunque $x_k\to \infty$, $L_f(k)-1/k<f(x_k)\le L_f(k)$ da cui si ricava
che $f(x_k)\to \limsup f$. Lo stesso si pu\`o fare per il $\liminf$.

Abbiamo dunque dimostrato il seguente teorema.
Se $x_k\to \infty$ e $f(x_k)\to l$ allora $l\in [\liminf f,\limsup
  f]$. Inoltre l'intervallo $[\liminf f,\limsup f]$ \`e il pi\`u
  piccolo intervallo con questa propriet\`a in quanto esistono sempre
  due sottosuccessioni di $f(x)$ che convergono agli estremi di tale
  intervallo.

Ricordando ora che $\lim f(x)=l$ se e solo se per ogni successione
$x_k\to \infty$ si ha $f(x_k)\to l$ ne deduciamo che vale il seguente teorema.

Il limite $\lim f$ esiste se e solo se $\liminf f = \limsup f$. E
in tal caso si ha $\lim f=\liminf f=\limsup f$.

\section*{Propriet\`a}
\[
\limsup f+g \le \limsup f + \limsup g,\quad
\limsup f\cdot g \le \limsup f \cdot \limsup g,
\]\[
\liminf f\cdot g \ge \liminf f \cdot \liminf g,\quad
\liminf f+g \ge \liminf f + \liminf g,\quad
\]\[
\limsup -f = -\liminf f,\quad
\limsup |f|=0 \Rightarrow \lim f=0.
\]

Dimostriamo ad esempio che $\liminf fg\ge \liminf f\liminf g$.
Sia $x_k$ tale che $f(x_k)g(x_k)\to \liminf fg$. Scelto comunque
$\epsilon>0$ definitivamente in $k$ vale $f(x_k)\ge \liminf
f-\epsilon$ e $g(x_k)\ge \liminf g -\epsilon$. Dunque $f(x_k)g(x_k)
\ge \liminf f \liminf g - \epsilon (\liminf f + \liminf g-\epsilon)$. Questo \`e
vero definitivamente in $k$ e quindi $\liminf fg = \lim_k
f(x_k)g(x_k)\ge \liminf f \liminf g - \epsilon (\cdots)$. Ma questo
\`e vero per ogni $\epsilon>0$ da cui la tesi.

\section*{Esempi}
Si ha $\limsup \sin(x)=1$, $\liminf \sin(x)=-1$ infatti $L(x)=1$ e
$l(x)=-1$. Notiamo poi che $\sin(2k\pi+\theta)\to \sin\theta$ e
quindi, in questo caso, ogni valore tra $-1$ e $1$
\`e un possibile limite di successioni del tipo $\sin x_k$.

Sia $f(x)$ la funzione indicatrice dei numeri razionali, ovvero
$f(x)=1$ se $x$ \`e razionale, $f(x)=0$ se x \`e irrazionale. Anche in
questo caso si mostra facilmente che $\limsup f =1$ e $\liminf f
=0$. Notiamo per\`o che ogni sottosuccessione di $f(x)$ non pu\`o
converge ad altri valori se non $0$ e $1$.

Si verica che $\limsup \sin(x)=1$, $\limsup \cos(x)=1$ ma $\limsup
\sin(x) + \cos(x) = \limsup 2(\sin(x+\pi/4))/\sqrt{2} = 2/\sqrt{2}<2$.

Dimostriamo che $\lim (\sin x) /x =0$ (sempre per $x\to
\infty$). Si ha infatti
\[
  \limsup \vert (\sin x) /x\vert \le \limsup \vert \sin x\vert \cdot
\limsup 1/x = 1 \cdot 0 = 0. 
\] 






\end{document}
