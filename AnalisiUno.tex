\documentclass[italian,a4paper,oneside,headinclude]{scrbook}
\usepackage{amsmath,amssymb,amsthm,thmtools}
\usepackage{babel,a4}
\usepackage[nochapters,pdfspacing]{classicthesis}
\usepackage[utf8]{inputenc}
\usepackage{comment} % for the comment environment
\usepackage{graphicx}
\usepackage{tikz}
\usepackage{pst-node}
\usetikzlibrary{cd} % commutative diagrams
\usepackage{eucal}
\AfterPreamble{\hypersetup{hidelinks=true,}}

%\usepackage{showkeys}

\usetikzlibrary{arrows}


\newcommand{\eps}{\varepsilon}
\renewcommand{\phi}{\varphi}
\newcommand{\loc}{\mathit{loc}}
\newcommand{\weakto}{\rightharpoonup}

% calligraphic letters
\newcommand{\A}{\mathcal A}
\newcommand{\B}{\mathcal B}
\newcommand{\C}{\mathcal C}
\newcommand{\D}{\mathcal D}
\newcommand{\E}{\mathcal E}
\newcommand{\F}{\mathcal F}
\newcommand{\FL}{\mathcal F\!\mathcal L}
\renewcommand{\L}{\mathcal L}
\newcommand{\M}{\mathcal M}
\renewcommand{\P}{\mathcal P}
\renewcommand{\S}{\mathcal S}

% blackboard letters
\newcommand{\CC}{\mathbb C}
\newcommand{\HH}{\mathbb H}
\newcommand{\KK}{\mathbb K}
\newcommand{\NN}{\mathbb N}
\newcommand{\QQ}{\mathbb Q}
\newcommand{\RR}{\mathbb R}
\newcommand{\TT}{\mathbb T}
\newcommand{\ZZ}{\mathbb Z}

\newcommand{\abs}[1]{{\left|#1\right|}}
\newcommand{\Abs}[1]{{\left\Vert #1\right\Vert}}

\renewcommand{\vec}[1]{\mathbf #1}
\newcommand{\defeq}{:=}
\DeclareMathOperator{\spt}{spt}
\DeclareMathOperator{\divergence}{div}
\renewcommand{\div}{\divergence}
% \DeclareMathOperator{\ker}{ker}  %% already defined
\DeclareMathOperator{\Imaginarypart}{Im}
\renewcommand{\Im}{\Imaginarypart}
\DeclareMathOperator{\Realpart}{Re}
\renewcommand{\Re}{\Realpart}

\newtheorem{theorem}{Teorema}
\newtheorem{lemma}[theorem]{Lemma}
\newtheorem{exercise}[theorem]{Esercizio}
\newtheorem{proposition}[theorem]{Proposizione}
\newtheorem{corollary}[theorem]{Corollario}
\newtheorem{example}[theorem]{Esempio}
\newtheorem{definition}[theorem]{Definizione}
\newtheorem{axiom}[theorem]{Assioma}

\title{Appunti di Analisi Matematica I%
\thanks{%
Puoi scaricare o contribuire a questi appunti su
\url{https://github.com/paolini/appunti/}}}
\author{E. Paolini}

\begin{document}
\maketitle

% \tableofcontents

\chapter{I numeri reali}

Supponiamo esista un insieme $\RR$ su cui sono definite le operazioni $+$ e $\cdot$
e la relazione d'ordine $\le$ che soddisfano i seguenti assiomi.

\begin{axiom}[campo]\label{axiom_field}
Sull'insieme $\RR$ dei numeri reali sono definite le operazioni di somma $+$ e
prodotto $\cdot$ che soddisfano le proprietà:
\begin{enumerate}
\item associativa: $(x+y)+z = x + (y+z)$, $(x\cdot y)\cdot z = x \cdot (x \cdot z)$);
\item commutativa: $x+y=y+x$, $x\cdot y = y \cdot x$;
\item distributiva: $x\cdot (y+z) = x\cdot y + x \cdot z$;
\item esistenza degli elementi neutri: $0,1\in \RR$,
$0\neq 1$, $0+x = x$, $1\cdot x = x$;
\item esistenza di opposto: per ogni $x$ esiste $y$ tale che $x+y = 0$;
\item esistenza del reciproco: per ogni $x\neq 0$ esiste $y$ tale che $x \cdot y = 1$.
\end{enumerate}
\end{axiom}

Denotiamo con $-x$ l'opposto di $x$ e definiamo $x-y = x+(-y)$.
Denotiamo con $y^{-1}$ il reciproco di $y\neq 0$ e
definiamo $x / y = x\cdot y^{-1}$.

\begin{axiom}[totalmente ordinato]\label{axiom_order}
Su $\RR$ è definita una relazione $\le$ con le seguenti proprietà
\begin{enumerate}
\item dicotomia: $x \le y$ o $y \le x$;
\item riflessiva: $x \le x$;
\item antisimmetrica: se $ x\le y$ e $y \le x$ allora $x=y$;
\item transitiva: se $x\le y $ e $ y \le z$ allora $x\le z$.
\end{enumerate}
\end{axiom}

Definiamo $x<y$ se $x\le y$ e $x \neq y$ e definiamo le relazioni
inverse $x \ge y$ se $y\le x$ e $x>y$ se $y<x$.

\begin{axiom}[campo ordinato]
Le operazioni di campo e l'ordinamento sono compatibili nel senso che
valgono le seguenti proprietà:
\begin{enumerate}
\item positività: se $x\ge 0$ e $y \ge 0$ allora $x+y \ge 0$ e $x\cdot y\ge 0$;
\item monotonia: se $x \ge y$ allora $x+z \ge y+z$.
\end{enumerate}
\end{axiom}

\begin{axiom}[completezza]\label{axiom_complete}
Se $A$ e $B$ sono sottoinsiemi non vuoti di $\RR$ tali che $A \le B$
(cioè: per ogni $a \in A$ e per ogni $b\in B$ vale $a\le b$) allora esiste
$x\in \RR$ tale che $A\le x \le B$ (cioè per ogni $a\in A$ e per ogni $b\in B$
vale $a\le x \le b$).
\end{axiom}


\begin{theorem}
In un campo ordinato valgono le seguenti
familiari proprietà:
\begin{enumerate}
  \item l'opposto e il reciproco sono unici (denotiamo con $-x$ l'unico opposto di $x$ e con $x^{-1}$ l'unico inverso di $x\neq 0$)
  \item $-(-x) = x$, $\left(x^{-1}\right)^{-1}$
  \item $x \cdot 0 = 0$
  \item $x\ge 0 \iff -x \le 0$
  \item $(-x)\cdot y = -(x\cdot y)$
  \item $-x = (-1)\cdot x$
  \item $(-1)\cdot(-1) = 1$
  \item $x\cdot x \ge 0$
  \item $1 > 0$
  \item se $x>0$ e  $y>0$  allora $x\cdot y > 0$
  \item se $x\cdot y = 0$ allora $x = 0$ o $y = 0$
\end{enumerate}
\end{theorem}
%
\begin{proof}
Supponiamo $y$ e $z$ siano due opposti di $x$ cioè $x+y=0$, $x+z=0$.
Allora da un lato $x+y+z = 0+z = z$, dall'altro $x+y+z = y+x+z= y+ 0 = y$.
Dunque $y=z$. Dimostrazione analoga si può fare per il reciproco.

Se $x$ è opposto di $y$ allora $y$ è opposto di $x$ in quanto la somma
è commutativa. Dunque l'opposto di $-x$ è $x$ cioè $-(-x)=x$. Lo stesso
vale per il reciproco.

Si ha
\[
x\cdot 0 = x \cdot 0 + x + (-x) %= x\cdot 0 + x\cdot 1 + (-x)
=x\cdot(0+1) + (-x) = x + (-x) = 0.
\]

Se $x\ge 0$ sommando ad ambo i membri $-x$ si ottiene $x+(-x) \ge 0 + (-x)$
cioè $0 \ge -x$. Sommando $x$ ad ambo i membri si riottiene $x\ge 0$.


Osserviamo che $(-x)\cdot y + x\cdot y = ((-x)+x)\cdot y = 0$ dunque $(-x)\cdot y$ è l'opposto di $x\cdot y$.
Dunque $(-1)\cdot x = - (1 \cdot x) = - x$.
E per $x=-1$ si ottiene $(-1)\cdot(-1) = -(-1) = 1$.

Si ha
\[
(-x)\cdot(-x) = (-1)\cdot x \cdot (-1)\cdot x = x\cdot x.
\]
Dunque se $x\ge 0$ per assioma di positività
abbiamo $x\cdot x\ge 0$ e se $x\le 0$ abbiamo $-x\ge 0$ e quindi
$x\cdot x = (-x)\cdot(-x) \ge 0$.
In particolare per $x=1$ otteniamo $1\ge 0$.
Essendo inoltre per assioma $0\neq 1$ otteniamo $1> 0$.

Se fosse $x\cdot y = 0$ e $x\neq 0$ allora $x$ avrebbe inverso $x^{-1}$
e avremmo:
\[
  y = x^{-1} \cdot x \cdot y = x^{-1}\cdot 0 = 0.
\]
Dunque o $x=0$ oppure $y=0$.

\end{proof}

\begin{definition}[numeri naturali]
Un sottoinsieme $A\subset \RR$ si dice essere \emph{induttivo}
se $0\in A$ e $n\in A \implies n+1 \in A$.
La famiglia di tutti i sottoinsiemi induttivi di $\RR$ non è vuota
in quanto $\RR$ stesso è induttivo. Definiamo $\NN$ come l'intersezione
di tutti i sottoinsiemi induttivi di $\RR$ (ovvero: il più piccolo sottoinsieme induttivo di $\RR$).
\end{definition}

Non è difficile dimostrare che l'insieme $\NN$ così definito
soddisfa gli assiomi di Peano (si vedano gli appunti di logica).

A partire da $\NN$ si può costruire l'insieme $\ZZ$ dei numeri interi
e l'insieme $\QQ$ dei numeri razionali. Si avrà dunque $\NN \subset \ZZ \subset \QQ \subset \RR$.
Si può verificare che $\QQ$ è un campo ordinato che però non soddisfa l'assioma di completezza.
Ad esempio i seguenti due insiemi:
\[
  A= \{x\in \QQ\colon x\ge 0, x^2 < 2\},
  B = \{x\in \QQ\colon x\ge 0, x^2 > 2\}
\]
sono insiemi non vuoti con la proprietà $A\le B$.
Non esiste nessun $x\in \QQ$ tale che $A\le x$ e $x\le B$.
Si può infatti dimostrare (ma lo faremo più avanti, quando avremo il concetto di \emph{continuità}) che se tale $x$ esistesse dovrebbe
soddisfare l'equazione $x^2=2$ e questo non è possibile per il seguente
teorema.

\begin{theorem}[Pitagora]
L'equazione $x^2=2$ non ha soluzioni in $\QQ$.
\end{theorem}
%
\begin{proof}
Supponiamo $x\in \QQ$ sia una soluzione di $x^2=2$.
Allora si potrà scrivere $x=p/q$ con $p\in \ZZ$ e $q\in \NN$, $q\neq 0$.
Possiamo anche supporre che la frazione $p/q$ sia ridotta ai minimi
termini cioè che $p$ e $q$ non abbiano fattori in comune.
Moltiplicando l'equazione
$(p/q)^2=2$ per $q^2$ si ottiene $p^2 = 2 q^2$.
Risulta quindi che $p^2$ è pari.
Ma allora anche $p$ è pari (perché il quadrato di un dispari è dispari).
Ma se $p$ è pari allora $p^2$ è multiplo di quattro.
Ma allora anche $2q^2$ è multiplo di quattro e quindi $q^2$ è pari.
Dunque anche $q$ è pari. Ma avevamo supposto che $p$ e $q$ non avessero
fattori in comune quindi questo non può accadere.
\end{proof}

\begin{definition}
Siano $x \in \RR$ e $A \subset \RR$. Se $A\le x$ (ovvero $a\le x$ per ogni $a\in A$)
diremo che $x$ è un \emph{maggiorante} di $A$.
Se $x \le A$ diremo invece che $x$ è un \emph{minorante} di $A$.
Se $A$ ammette un maggiorante diremo che $A$ è \emph{superiormente limitato},
se $A$ ammette un minorante diremo che $A$ è \emph{inferiormente limitato},
se $A$ ammette sia maggiorante che minorante diremo che $A$ è \emph{limitato}.

Se $A \le x$ e $x\in A$ diremo che $x$ è un massimo di $A$,
se $x\le A$ e $x\in A$ diremo che $x$ è un minimo di $A$

Se $x$ è il minimo dei maggioranti di $A$ diremo che $x$ è
un \emph{estremo superiore}
di $A$ se invece $x$ è il massimo dei minoranti diremo che $x$ è
un \emph{estremo inferiore} di $A$.
\end{definition}

Il massimo e il minimo di un insieme $A$, se esistono, sono unici.
Infatti se $x$ e $y$ fossero due massimi di $A$ si avrebbe $x\le y$ in
quanto $x\le A$ e $y\in A$. Analogamente si avrebbe $y\le x$ e
quindi $x=y$.
Anche l'estremo superiore e l'estremo inferiore se esistono sono
unici in quanto sono essi stessi un minimo ed un massimo
(rispettivamente dei maggioranti e dei minoranti).

Useremo quindi le notazioni:
\[
 \max A, \qquad
 \min A, \qquad
 \sup A, \qquad
 \inf A
\]
per denotare univocamente (quando esistono) il massimo, il minimo,
l'estremo superiore e l'estremo inferiore di un insieme $A$.

\begin{theorem}
Se $A$ è un insieme non vuoto,
superiormente limitato, allora esiste l'estremo superiore di $A$.
Tale numero $x=\sup A$ è caratterizzato dalle seguenti proprietà
\begin{enumerate}
\item $\forall a\in A\colon x \ge a$;
\item $\forall \eps>0 \exists a\in A \colon x < a + \eps$.
\end{enumerate}

Risultato analogo vale per l'estremo inferiore.
\end{theorem}
%
\begin{proof}
Consideriamo l'insieme dei maggioranti
$B = \{ b\in \RR \colon b \ge A\}$.
Per ipotesi $B$ è non vuoto e per come è definito risulta $A\le B$.
Dunque dall'assioma di completezza deduciamo l'esistenza di un numero $x\in \RR$
tale che $A\le x \le B$. La prima disuguaglianza $A\le x$ ci dice che $x$ è un
maggiorante e quindi $x\in B$, la seconda $x\le B$ ci dice che $x$ è il minimo
di $B$ e quindi concludiamo che $x$ è l'estremo superiore di $A$.
La prima delle due proprietà caratterizzanti il $\sup$ traduce la condizione
che $x$ sia un maggiorante di $A$. La seconda delle due proprietà esprime il
fatto che $x$ sia il minimo dei maggioranti, infatti se $x$ è il minimo
dei maggioranti significa che nessun numero minore di $x$ è un maggiorante, ovvero
che ogni $x-\eps$ con $\eps>0$ non è un maggiorante, ovvero
che esiste $a\in A$ tale che $a > x-\eps$.
\end{proof}

E' molto comodo avere gli operatori $\sup$ e $\inf$ che si applicando a qualunque
insieme limitato e non vuoto. Ancora più comodo sarebbe poter applicare tali operatori
indistintamente ad qualunque sottoinsieme di $\RR$. Per fare ciò consideriamo la seguente
estensione dei numeri reali.

\begin{definition}[reali estesi]
Denotiamo con $\bar \RR=\RR \cup \{+\infty, -\infty\}$ l'insieme dei numeri reali
a cui vengono aggiunti due ulteriori \emph{quantità} che chiameremo
\emph{infinite} e che denotiamo con $+\infty$ e $-\infty$.
Estendiamo la relazione d'ordine imponendo che valga
\[
  -\infty < x < +\infty, \qquad \forall x \in \RR.
\]

Estendiamo anche la somma e il prodotto tra reali estesi imponendo che valga
per ogni $x\in \RR$
\[
  x + \infty = +\infty, \qquad
  x - \infty = -\infty, \qquad
  +\infty + \infty = +\infty, \qquad
  -\infty -\infty = -\infty
\]
e per ogni $x > 0$:
\[
  x \cdot (+\infty) = +\infty, \qquad
  x \cdot (-\infty) = -\infty, \qquad
  +\infty \cdot (+\infty) = +\infty, \qquad
  +\infty \cdot (-\infty) = -\infty
\]
e definizioni analoghe (con i segni cambiati) per $x<0$.
Possiamo anche definire per ogni $x\in \RR$
\[
x / (+\infty) = x / (-\infty) = 0.
\]

Si lasciano di norma  indefinite
le operazioni $+\infty - \infty$, $0 \cdot +\infty$ e divisione tra infiniti.
\end{definition}
% \bibliographystyle{plain}
% \bibliography{biblio}

\begin{theorem}[proprietà archimedea]
Dato $x\in \RR$ esiste $n\in \NN$ tale che $n > x$.
E se $x>0$ esiste $m\in \NN$ tale che $1/m < x$.
\end{theorem}
%
\begin{proof}

\end{proof}


\end{document}
