\documentclass[italian,a4paper,oneside,headinclude]{scrbook}
\usepackage{amsmath,amssymb,amsthm,thmtools}
\usepackage{babel,a4}
\usepackage[nochapters,pdfspacing]{classicthesis}
\usepackage[utf8]{inputenc}
\usepackage{comment} % for the comment environment
\usepackage{graphicx}
\usepackage{tikz}
\usepackage{pst-node}
\usetikzlibrary{cd} % commutative diagrams
\usepackage{eucal}
\AfterPreamble{\hypersetup{hidelinks=true,}}

%\usepackage{showkeys}

\usetikzlibrary{arrows}

\newcommand{\myemph}[1]{\emph{#1}\marginpar{#1}}
\renewcommand{\subset}{\subseteq}

\newcommand{\eps}{\varepsilon}
\renewcommand{\phi}{\varphi}
\newcommand{\loc}{\mathit{loc}}
\newcommand{\weakto}{\rightharpoonup}

% calligraphic letters
\newcommand{\A}{\mathcal A}
\newcommand{\B}{\mathcal B}
\newcommand{\C}{\mathcal C}
\newcommand{\D}{\mathcal D}
\newcommand{\E}{\mathcal E}
\newcommand{\F}{\mathcal F}
\newcommand{\FL}{\mathcal F\!\mathcal L}
\renewcommand{\L}{\mathcal L}
\newcommand{\M}{\mathcal M}
\renewcommand{\P}{\mathcal P}
\renewcommand{\S}{\mathcal S}

% blackboard letters
\newcommand{\CC}{\mathbb C}
\newcommand{\HH}{\mathbb H}
\newcommand{\KK}{\mathbb K}
\newcommand{\NN}{\mathbb N}
\newcommand{\QQ}{\mathbb Q}
\newcommand{\RR}{\mathbb R}
\newcommand{\TT}{\mathbb T}
\newcommand{\ZZ}{\mathbb Z}

\newcommand{\abs}[1]{{\left|#1\right|}}
\newcommand{\Abs}[1]{{\left\Vert #1\right\Vert}}

\renewcommand{\vec}[1]{\mathbf #1}
\newcommand{\defeq}{:=}
\DeclareMathOperator{\spt}{spt}
\DeclareMathOperator{\divergence}{div}
\renewcommand{\div}{\divergence}
% \DeclareMathOperator{\ker}{ker}  %% already defined
\DeclareMathOperator{\Imaginarypart}{Im}
\renewcommand{\Im}{\Imaginarypart}
\DeclareMathOperator{\Realpart}{Re}
\renewcommand{\Re}{\Realpart}
%\DeclareMathOperator{\arg}{arg}

\newtheorem{theorem}{Teorema}
\newtheorem{lemma}[theorem]{Lemma}
\newtheorem{exercise}[theorem]{Esercizio}
\newtheorem{proposition}[theorem]{Proposizione}
\newtheorem{corollary}[theorem]{Corollario}
\newtheorem{example}[theorem]{Esempio}
\newtheorem{definition}[theorem]{Definizione}
\newtheorem{axiom}[theorem]{Assioma}

\title{Appunti di Analisi Matematica I%
\thanks{%
Puoi scaricare o contribuire a questi appunti su
\url{https://github.com/paolini/appunti/}}}
\author{E. Paolini}

\begin{document}
\maketitle

% \tableofcontents

\chapter{I numeri reali}

Supponiamo esista un insieme $\RR$ su cui sono definite le operazioni $+$ e $\cdot$
e la relazione d'ordine $\le$ che soddisfano i seguenti assiomi.

\begin{axiom}[campo]\label{axiom_field}
Sull'insieme $\RR$ dei numeri reali sono definite le operazioni di somma $+$ e
prodotto $\cdot$ che soddisfano le proprietà:
\begin{enumerate}
\item associativa: $(x+y)+z = x + (y+z)$, $(x\cdot y)\cdot z = x \cdot (x \cdot z)$);
\item commutativa: $x+y=y+x$, $x\cdot y = y \cdot x$;
\item distributiva: $x\cdot (y+z) = x\cdot y + x \cdot z$;
\item esistenza degli elementi neutri: $0,1\in \RR$,
$0\neq 1$, $0+x = x$, $1\cdot x = x$;
\item esistenza di opposto: per ogni $x$ esiste $y$ tale che $x+y = 0$;
\item esistenza del reciproco: per ogni $x\neq 0$ esiste $y$ tale che $x \cdot y = 1$.
\end{enumerate}
\end{axiom}

Denotiamo con $-x$ l'opposto di $x$ e definiamo $x-y = x+(-y)$.
Denotiamo con $y^{-1}$ il reciproco di $y\neq 0$ e
definiamo $x / y = x\cdot y^{-1}$.

\begin{axiom}[totalmente ordinato]\label{axiom_order}
Su $\RR$ è definita una relazione $\le$ con le seguenti proprietà
\begin{enumerate}
\item dicotomia: $x \le y$ o $y \le x$;
\item riflessiva: $x \le x$;
\item antisimmetrica: se $ x\le y$ e $y \le x$ allora $x=y$;
\item transitiva: se $x\le y $ e $ y \le z$ allora $x\le z$.
\end{enumerate}
\end{axiom}

Definiamo $x<y$ se $x\le y$ e $x \neq y$ e definiamo le relazioni
inverse $x \ge y$ se $y\le x$ e $x>y$ se $y<x$.

\begin{axiom}[campo ordinato]
Le operazioni di campo e l'ordinamento sono compatibili nel senso che
valgono le seguenti proprietà:
\begin{enumerate}
\item positività: se $x\ge 0$ e $y \ge 0$ allora $x+y \ge 0$ e $x\cdot y\ge 0$;
\item monotonia: se $x \ge y$ allora $x+z \ge y+z$.
\end{enumerate}
\end{axiom}

\begin{axiom}[completezza]\label{axiom_complete}
Se $A$ e $B$ sono sottoinsiemi non vuoti di $\RR$ tali che $A \le B$
(cioè: per ogni $a \in A$ e per ogni $b\in B$ vale $a\le b$) allora esiste
$x\in \RR$ tale che $A\le x \le B$ (cioè per ogni $a\in A$ e per ogni $b\in B$
vale $a\le x \le b$).
\end{axiom}


\begin{theorem}
In un campo ordinato valgono le seguenti
familiari proprietà:
\begin{enumerate}
  \item l'opposto e il reciproco sono unici (denotiamo con $-x$ l'unico opposto di $x$ e con $x^{-1}$ l'unico inverso di $x\neq 0$)
  \item $-(-x) = x$, $\left(x^{-1}\right)^{-1}$
  \item $x \cdot 0 = 0$
  \item $x\ge 0 \iff -x \le 0$
  \item $(-x)\cdot y = -(x\cdot y)$
  \item $-x = (-1)\cdot x$
  \item $(-1)\cdot(-1) = 1$
  \item $x\cdot x \ge 0$
  \item $1 > 0$
  \item se $x\cdot y = 0$ allora $x = 0$ o $y = 0$
  \item se $x>0$ e $y>0$  allora $x\cdot y > 0$
\end{enumerate}
\end{theorem}
%
\begin{proof}
\begin{enumerate}
\item
Supponiamo $y$ e $z$ siano due opposti di $x$ cioè $x+y=0$, $x+z=0$.
Allora da un lato $x+y+z = 0+z = z$, dall'altro $x+y+z = y+x+z= y+ 0 = y$.
Dunque $y=z$. Dimostrazione analoga si può fare per il reciproco.

\item
Se $x$ è opposto di $y$ allora $y$ è opposto di $x$ in quanto la somma
è commutativa. Dunque l'opposto di $-x$ è $x$ cioè $-(-x)=x$. Lo stesso
vale per il reciproco.

\item
Si ha
\[
x\cdot 0 = x \cdot 0 + x + (-x) %= x\cdot 0 + x\cdot 1 + (-x)
=x\cdot(0+1) + (-x) = x + (-x) = 0.
\]

\item
Se $x\ge 0$ sommando ad ambo i membri $-x$ si ottiene $x+(-x) \ge 0 + (-x)$
cioè $0 \ge -x$. Sommando $x$ ad ambo i membri si riottiene $x\ge 0$.


\item
Osserviamo che $(-x)\cdot y + x\cdot y = ((-x)+x)\cdot y = 0$ dunque $(-x)\cdot y$ è l'opposto di $x\cdot y$.

\item
Dunque $(-1)\cdot x = - (1 \cdot x) = - x$

\item
e per $x=-1$ si ottiene $(-1)\cdot(-1) = -(-1) = 1$.

\item
Si ha
\[
(-x)\cdot(-x) = (-1)\cdot x \cdot (-1)\cdot x = x\cdot x.
\]
Dunque se $x\ge 0$ per assioma di positività
abbiamo $x\cdot x\ge 0$ e se $x\le 0$ abbiamo $-x\ge 0$ e quindi
$x\cdot x = (-x)\cdot(-x) \ge 0$.

\item
In particolare per $x=1$ otteniamo $1\ge 0$.
Essendo inoltre per assioma $0\neq 1$ otteniamo $1> 0$.

\item
Se fosse $x\cdot y = 0$ e $x\neq 0$ allora $x$ avrebbe inverso $x^{-1}$
e avremmo:
\[
  y = x^{-1} \cdot x \cdot y = x^{-1}\cdot 0 = 0.
\]
Dunque o $x=0$ oppure $y=0$.

\item
Se $x>0$ e $y>0$ allora $x\ge 0$ e $y\ge 0$ da cui $x\cdot y\ge 0$.
Se fosse $x\cdot y=0$ uno dei due fattori si dovrebbe annullare
cosa che abbiamo escluso per ipotesi.
\end{enumerate}
\end{proof}

\begin{theorem}[radice quadrata]
Dato $y\ge 0$ esiste un unico $x\ge 0$ tale che $x^2=y$.
Tale $x$ verrà denotato con $\sqrt y$, \emph{radice quadrata} di $y$.
\end{theorem}
\begin{proof}
Se $y=0$ allora è facile verificare che $x^2=y$ ha come unica soluzione $x=0$.
Supponiamo allora $y>0$ e
consideriamo i seguenti due insiemi
\[
  A = \{x\ge 0 \colon x^2 \le y\},\qquad
  B = \{x\ge 0 \colon x^2 \ge y\}
\]
e verifichiamo che soddisfino le ipotesi dell'assioma di completeza.
Innanzitutto $0\in A$ e quindi $A$ non è vuoto.
Neanche $B$ è vuoto in quanto $y+1\in B$,
infatti essendo $y+1\ge 1$ si ha
$(y+1)^2 \ge y+1$. Verifichiamo inoltre che $A \le B$.
Preso $a\in A$ e $b\in B$ si ha $a^2 \le y \le b^2$.
Se fosse $a>b$ dovremmo avere $a^2>b^2$, dunque $a \le b$.

Dunque possiamo applicare l'assioma di completezza
che ci garantisce l'esistenza di $z\in \RR$ tale che $A \le z \le B$.
Vogliamo ora verificare che $z^2 = y$.

Ci servirà innanzitutto sapere che $z>0$. Se $y\ge 1$ si avrebbe $1\in A$
e dunque $z\ge 1$ essendo $z\ge A$. Se $y<1$ allora $y^2 < y$ e dunque $y^2 \in A$
da cui si ottiene $z\ge y^2 > 0$.

Se fosse $z^2 < y$ vorremmo dimostrare che esiste $\eps>0$ tale che
$(z+\eps)^2 \le y$.
Questo succede se $(z+\eps)^2 = z^2 + 2 \eps z + \eps^2 \le y$.
Questo si può ottenere, ad esempio,
imponendo che sia $2\eps z \le (y-z^2)/2$ e $\eps^2 \le (y-z^2)/2$.
Cioè (ricordiamo che $z>0$) se $\eps \le (y-z^2)/(4z)$ e $\eps \le 1$
(in modo che $\eps^2 \le \eps$)
e $\eps \le (y-z^2)/2$. Dunque scegliendo
\[
\eps = \min\left\{ \frac{y-z^2}{4z}, 1, \frac{y-z^2}/2\right\}
\]
si osserva che $\eps>0$ e vale $(z+\eps)^2\le y$.
Dunque $z+\eps \in A$ e dunque non può essere $z\ge A$.

Se fosse $z^2 > y$ vorremmo dimostrare che esiste $\eps>0$ tale che
$(z-\eps)^2 \ge y$.
Questo succede se $(z-\eps)^2 = z^2 - 2\eps z + \eps^2 \le y$.
E' quindi sufficiente che sia $z^2 - 2 \eps z \le y$ ovvero basta scegliere
\[
  \eps = \frac{y-z^2}{2z}.
\]
Ma allora se $(z-\eps)^2\ge y$ si ha $z-\eps \in B$ e dunque non può
essere $z \le B$.

Rimane dunque l'unica possibilità che sia $z^2 = y$, come volevamo dimostrare.

Se ci fosse un altro $w\ge 0$ tale che $w^2 = y$ si avrebbe $w^2 - z^2=0$ ovvero
$(w-z)(w+z)=0$ da cui (ricordando che $z>0$ e quindi $w+z\neq 0$)
si ottiene $w-z=0$. Dunque $w=z$.
\end{proof}


\section{i numeri naturali, interi e razionali}

\begin{definition}[numeri naturali]
Un sottoinsieme $A\subset \RR$ si dice essere \emph{induttivo}
se $0\in A$ e $n\in A \implies n+1 \in A$.
La famiglia di tutti i sottoinsiemi induttivi di $\RR$ non è vuota
in quanto $\RR$ stesso è induttivo. Definiamo $\NN$ come l'intersezione
di tutti i sottoinsiemi induttivi di $\RR$ (ovvero: il più piccolo sottoinsieme induttivo di $\RR$).
\end{definition}

Non è difficile dimostrare che l'insieme $\NN$ così definito
soddisfa gli assiomi di Peano (si vedano gli appunti di logica).

A partire da $\NN$ si può costruire l'insieme $\ZZ$ dei numeri interi
e l'insieme $\QQ$ dei numeri razionali:
\begin{align*}
  \ZZ
    &= \NN \cup (-\NN)
    = \{x\in \RR\colon \exists n\in\NN\colon (x=n) \lor (x=-n)\}, \\
  \QQ
    &= \frac{\ZZ}{\NN\setminus\{0\}}
    = \left\{x \in \RR\colon \exists p\in \ZZ\colon \exists q \in \NN\setminus\{0\}\colon x = \frac{p}{q}\right\}
\end{align*}

Si avrà dunque $\NN \subset \ZZ \subset \QQ \subset \RR$.
Si può verificare che $\QQ$ è un campo ordinato che però non soddisfa l'assioma di completezza.

\begin{theorem}[Pitagora]
L'equazione $x^2=2$ non ha soluzioni in $\QQ$.
\end{theorem}
%
\begin{proof}
Supponiamo $x\in \QQ$ sia una soluzione di $x^2=2$.
Allora si potrà scrivere $x=p/q$ con $p\in \ZZ$ e $q\in \NN$, $q\neq 0$.
Possiamo anche supporre che la frazione $p/q$ sia ridotta ai minimi
termini cioè che $p$ e $q$ non abbiano fattori in comune.
Moltiplicando l'equazione
$(p/q)^2=2$ per $q^2$ si ottiene $p^2 = 2 q^2$.
Risulta quindi che $p^2$ è pari.
Ma allora anche $p$ è pari (perché il quadrato di un dispari è dispari).
Ma se $p$ è pari allora $p^2$ è multiplo di quattro.
Ma allora anche $2q^2$ è multiplo di quattro e quindi $q^2$ è pari.
Dunque anche $q$ è pari. Ma avevamo supposto che $p$ e $q$ non avessero
fattori in comune quindi questo non può accadere.
\end{proof}

\section{estremo superiore}

\begin{definition}
Siano $x \in \RR$ e $A \subset \RR$. Se $A\le x$ (ovvero $a\le x$ per ogni $a\in A$)
diremo che $x$ è un \emph{maggiorante} di $A$.
Se $x \le A$ diremo invece che $x$ è un \emph{minorante} di $A$.
Se $A$ ammette un maggiorante diremo che $A$ è \emph{superiormente limitato},
se $A$ ammette un minorante diremo che $A$ è \emph{inferiormente limitato},
se $A$ ammette sia maggiorante che minorante diremo che $A$ è \emph{limitato}.

Se $A \le x$ e $x\in A$ diremo che $x$ è il massimo di $A$,
se $x\le A$ e $x\in A$ diremo che $x$ è il minimo di $A$

Se $x$ è minimo dei maggioranti di $A$ diremo che $x$ è
\emph{estremo superiore}
di $A$ se invece $x$ è massimo dei minoranti diremo che $x$ è
\emph{estremo inferiore} di $A$.
\end{definition}

Massimo e minimo di un insieme $A$, se esistono, sono unici.
Infatti se $x$ e $y$ fossero due massimi di $A$ si avrebbe $x\le y$ in
quanto $x\le A$ e $y\in A$. Analogamente si avrebbe $y\le x$ e
quindi $x=y$.
Anche l'estremo superiore e l'estremo inferiore se esistono sono
unici in quanto sono essi stessi un minimo ed un massimo
(rispettivamente dei maggioranti e dei minoranti).

Useremo quindi le notazioni:
\[
 \max A, \qquad
 \min A, \qquad
 \sup A, \qquad
 \inf A
\]
per denotare univocamente (quando esistono) il massimo, il minimo,
l'estremo superiore e l'estremo inferiore di un insieme $A$.

\begin{theorem}
Se $A$ è un insieme non vuoto,
superiormente limitato, allora esiste l'estremo superiore di $A$.
Tale numero $x=\sup A$ è caratterizzato dalle seguenti proprietà
\begin{enumerate}
\item $\forall a\in A\colon x \ge a$;
\item $\forall \eps>0\colon \exists a\in A \colon x < a + \eps$.
\end{enumerate}

Risultato analogo vale per l'estremo inferiore.
\end{theorem}
%
\begin{proof}
Consideriamo l'insieme dei maggioranti
$B = \{ b\in \RR \colon b \ge A\}$.
Per ipotesi $B$ è non vuoto e per come è definito risulta $A\le B$.
Dunque dall'assioma di completezza deduciamo l'esistenza di un numero $x\in \RR$
tale che $A\le x \le B$. La prima disuguaglianza $A\le x$ ci dice che $x$ è un
maggiorante e quindi $x\in B$, la seconda $x\le B$ ci dice che $x$ è il minimo
di $B$ e quindi concludiamo che $x$ è l'estremo superiore di $A$.
La prima delle due proprietà caratterizzanti il $\sup$ traduce la condizione
che $x$ sia un maggiorante di $A$. La seconda delle due proprietà esprime il
fatto che $x$ sia il minimo dei maggioranti, infatti se $x$ è il minimo
dei maggioranti significa che nessun numero minore di $x$ è un maggiorante, ovvero
che ogni $x-\eps$ con $\eps>0$ non è un maggiorante, ovvero
che esiste $a\in A$ tale che $a > x-\eps$.
\end{proof}

La seguente proprietà dei numeri reali esprime il fatto
che non esistono gli \emph{infinitesimi} ovvero numeri reali positivi
che siano più piccoli di ogni $1/n$ con $n\in \NN$.

\begin{theorem}[proprietà archimedea dei numeri reali]
Dato $x\in \RR$ esiste $n\in \NN$ tale che $n > x$.
E se $x>0$ esiste $m\in \NN$ tale che $1/m < x$.
\end{theorem}
%
\begin{proof}
Se esistesse $x\in \RR$ tale che $x \ge \NN$
allora $\NN$ sarebbe superiormente limitato.
Dunque avrebbe un estremo superiore $m= \sup \NN$.
Siccome $m$ è il minimo dei maggioranti di $\NN$
e $m-1$ è più piccolo di $m$, allora $m-1$ non è un maggiorante
di $\NN$. Dunque deve esistere $n\in \NN$ tale che $n>m-1$.
Ma allora $n+1 > m$ ed essendo $n+1\in \NN$ troviamo che $m$
non poteva essere un maggiorante di $\NN$.

Dunque per ogni $y\in \RR$ esiste $n\in \NN$ tale che $n>y$.
Se $x\in \RR$ e $x>0$ allora posto $y=1/x$ possiamo trovare
$n\in \NN$ con $n>y = 1/x$ da cui $x > 1/n$.
\end{proof}

\begin{theorem}[parte intera]
Dato $x\in \RR$ esiste un unico $m\in \ZZ$ tale che $m-1 \le x < m$.
\end{theorem}
%
\begin{proof}
Supponiamo per un attimo che sia $x\ge 1$.
In tal caso consideriamo l'insieme $A=\{n\in \NN\colon n > x\}$.
Per la proprietà archimedea tale insieme non può essere vuoto e,
per il buon ordinamento di $\NN$ (si vedano gli appunti di logica),
deve avere un minimo $m$.
Dunque $m>x$ (in quanto $m\in a$) e $m\ge 1$ (in quanto $x\ge 1$).
Quindi necessariamente $m-1 \le x$ altrimenti avremmo che $m-1\in A$ e $m$
non poteva essere il minimo. Si ottiene dunque $m-1\le x < m$ come volevamo
dimostrare.

Nel caso fosse $x<1$ possiamo trovare un $N\in \NN$ (sempre per la proprietà archimedea) per cui $x+N \ge 1$. Applicando il ragionamento precedente a $x+N$ si trova comunque il risultato desiderato.
\end{proof}

\begin{definition}[parte intera]
Dato $x\in \RR$ denotiamo con $\lfloor x\rfloor$ l'unico intero
che soddisfa
\[
  x - 1 < \lfloor x \rfloor \le x
\]
e denotiamo con $\lceil x \rceil = - \lfloor -x \rfloor$ l'unico intero che soddisfa (verificare!)
\[
  x \le \lceil x \rceil \le x + 1.
\]
Si ha dunque
\[
  \lfloor x \rfloor \le x \le \lceil x \rceil
\]
con entrambe le uguaglianze che si realizzano quando $x\in \ZZ$.
I due interi $\lfloor x \rfloor$ e $\lceil x \rceil$
sono la migliore approssimazione intera di $x$ rispettivamente
per difetto e per eccesso.
L'intero più vicino ad $x$ (approssimazione per arrotondamento)
è
\[
  \left\lfloor x + \frac 1 2 \right\rfloor
\quad \text{ossia} \quad
  \left\lceil x-\frac 1 2 \right\rceil
\]
(le due espressioni differiscono solamente quando $x$ si trova nel punto medio tra due interi consecutivi, nel qual caso la prima approssima per eccesso e la seconda per difetto).
\end{definition}
In alcuni testi si usa la notazione $[x]$ per denotare la parte intera $\lfloor x \rfloor$ e si definisce
anche la \emph{parte frazionaria}
\[
   \{x\} = x - [x]
\]
noi preferiamo evitare queste notazioni che possono risultare ambigue.

\begin{theorem}[densità di $\QQ$ in $\RR$]
Dati $x,y \in \RR$ con $x<y$ esiste $q\in \QQ$ tale che $x<q<y$.
\end{theorem}
%
\begin{proof}
Per la proprietà archimedea dei numeri naturali essendo $y-x>0$
deve esistere $n\in \NN$ tale che $y-x > 1/n$ così si avrà
\[
    nx + 1 < ny.
\]
Prendiamo allora $m=\lfloor nx + 1\rfloor$ cosicché si abbia
\[
  nx < m \le nx + 1.
\]
Mettendo insieme le due disuguaglianze e dividendo per n si ottiene,
come volevamo dimostrare,
\[
 x < \frac{m}{n} < y.
\]
\end{proof}

\section{reali estesi}

\begin{definition}[reali estesi]
Denotiamo con $\bar \RR=\RR \cup \{+\infty, -\infty\}$ l'insieme dei numeri reali
a cui vengono aggiunti due ulteriori \emph{quantità} che chiameremo
\emph{infinite} e che denotiamo con $+\infty$ e $-\infty$.
\end{definition}


Estendiamo la relazione d'ordine imponendo che valga
\[
  -\infty \le x \le +\infty, \qquad \forall x \in \bar\RR.
\]

Estendiamo anche la addizione e moltiplicazione
tra reali estesi imponendo che valga per ogni $x\in \bar \RR$
\begin{gather*}
  x + (+\infty) = +\infty, \qquad \text{se $x\neq -\infty$}\\
  x + (-\infty) = -\infty, \qquad \text{se $x\neq +\infty$}\\
  x \cdot (+\infty) = +\infty, \qquad
  x \cdot (-\infty) = -\infty, \qquad \text{se $x>0$} \\
  x \cdot (+\infty) = -\infty, \qquad
  x \cdot (-\infty) = +\infty, \qquad \text{se $x<0$}.
\end{gather*}

Si definiscono anche:
\[
 -(+\infty) = -\infty, \qquad
 -(-\infty) = +\infty, \qquad
 \frac{1}{+\infty} = \frac{1}{-\infty}=0
\]
facendo però attenzione che
questi formalmente non sono \emph{opposto}
e \emph{reciproco} in quanto
su $\bar R$ non sono più garantite
le regole: $x + (-x) = 0$ e $x \cdot (1/x) = 1$.
Infatti
le operazioni $(+\infty) + (-\infty)$ e $+\infty \cdot 0$ vengono lasciate indefinite.

Possiamo infine definire la sottrazione e la divisione tramite
addizione e moltiplicazione:
\[
  x - y = x + (-y), \qquad \frac{x}{y} = x \cdot \frac{1}{y}.
\]

Possiamo definire gli operatori $\sup$ e $\inf$
anche sugli insiemi illimitati ponendo:
\begin{align*}
  \sup A = +\infty \qquad \text{se $A$ non è superiormente limitato}\\
  \inf A = -\infty \qquad \text{se $A$ non è inferiormente limitato}
\end{align*}
e infine
\begin{align*}
  \sup \emptyset = -\infty\\
  \inf \emptyset = +\infty.
\end{align*}
Osserviamo infatti che su $\bar \RR$ la quantità $+\infty$
è maggiorante di qualunque insieme e $-\infty$ è minorante, dunque
queste definizioni mantengono su $\bar \RR$ le proprietà caratterizzanti:
l'estremo superiore è il minimo dei maggioranti e
l'estremo inferiore è il massimo dei minoranti.

\section{intervalli}

\begin{definition}[intervallo]
Un insieme $I\subset \RR$ si dice essere un \emph{intervallo}
se soddisfa la \emph{proprietà dei valori intermedi}:
\[
  \text{se $x, y \in I$ e $x<z<y$ allora $z \in I$.}
\]
\end{definition}
\begin{theorem}[caratterizzazione intervalli]
Sia $I$ un intervallo e sia $a=\inf I$, $b=\sup I$. Allora
$z\in I$ se $a < z < b$.
\end{theorem}
%
\begin{proof}
Se $I=\emptyset$ si ha $a>b$ e quindi nessun $z$ verifica $a<z<b$.
Supponiamo $I\neq \emptyset$ e
sia $a < z < b$.
Visto che $a$ è il massimo dei minoranti di $I$ deve esistere $x \in I$ tale
che $a \le x < z$ altrimenti ogni $x$ tra $a$ e $z$ sarebbe un minorante di $I$
e $a$ non sarebbe il minimo. Analogamente dovrebbe esistere $y\in I$ con $z<y\le b$.
Ma allora, per la proprietà dei valori intermedi anche $z\in I$.
\end{proof}

Il teorema precedente ci dice che una volta identificati i due estermi
di un intervallo, tutti i punti intermedi devono stare nell'intervallo.
Gli estremi, invece, possono essere o non essere inclusi nell'intervallo.
Punti esterni agli estremi non possono invece esserci.
Possiamo quindi caratterizzare tutti gli intervalli di $\bar \RR$
introducendo le seguenti notazioni. Dati $a,b\in \bar \RR$ con $a\le b$
poniamo
\begin{align*}
[a,b] &= \{x\in \bar \RR\colon a \le x \le b\} \\
[a,b) &= \{x\in \bar \RR\colon a \le x < b\} \\
(a,b] &= \{x\in \bar \RR\colon a < x \le b\}\\
(a,b) &= \{x\in \bar \RR\colon a < x < b\}.
\end{align*}
Abbiamo quindi utilizzato le parentesi quadre per indicare che gli estremi
sono inclusi e le parentesi tonde per indicare che gli estremi sono esclusi.
Osserviamo che in alcuni testi si usano le parentesi quadre rovesciate al posto
delle parentesi tonde.

Noi considereremo per lo più intervalli di $\RR$ (non di $\bar \RR$) e in tal
caso gli estremi infiniti non potranno mai essere inclusi.

\section{i numeri complessi}

Dal punto di vista geometrico l'insieme $\CC$ dei \emph{numeri complessi}
è un modello del piano euclideo.
Il piano euclideo è uno spazio affine reale di dimensione 2.
Possiamo mettere delle coordinate sul piano se fissiamo un punto $O$ (origine)
e due vettori ortonormali $e_1$, $e_2$. Chiamiamo $0$ il vettore
nullo $OO$ e chiamiamo $1$ il vettore $e_1$.
La retta passante per $O$ con direzione $e_1$ rappresenta i numeri reali
$\RR$ come abbiamo già visto. Chiamiamo $i$ il vettore $e_2$.
La retta passante per $O$ con direzione $e_2$ verrà chiamata
\emph{retta dei numeri immaginari}.

Un generico punto del piano $z$ potrà essere scritto nella base scelta in maniera
univoca: $z = x e_1 + y e_2$. Per come abbiamo definito $e_1$ ed
$e_2$ scriveremo più semplicemente $z = x + i y$. Tale $z$ viene chiamato
\emph{numero complesso} con parte reale $x$ e parte immaginaria $y$.
I numeri reali sono \emph{immersi} nei complessi, nel senso che se
$x\in \RR$ allora $z= x + i\cdot 0 = x$ è anche un numero complesso.
Il numero complesso $i = 0 + i\cdot 1$ viene chiamata \emph{unità immaginaria}
e i numeri complessi della forma $iy$ sono chiamati \emph{immaginari}.
Un numero
complesso $z = x+iy$ è quindi una somma tra un numero reale ed un numero
immaginario. Il numero reale $x$ viene chiamato \emph{parte reale} di $z$ e
si denota a volte con $x=\Re z$. Il numero reale $y$ viene chiamato
\emph{parte immaginaria} di $z$ e si denota con $y=\Im z$
(osserviamo che la parte immaginaria di un numero complesso è un numero
reale, non immaginario). Dunque $z= \Re z + i \Im z$.

L'insieme dei numeri complessi viene denotato con $\CC$.
Lo spazio $\CC$, per come
è stato costruito, è uno spazio vettoriale reale di dimensione $2$.
Abbiamo quindi già definite la somma tra elementi di $\CC$ e la moltiplicazione
tra elementi di $\CC$ ed elementi di $\RR$.
Se $a,b,c,d,t\in \RR$ si ha:
\[
 (a+ib) + (c+id) = (a+b) + i (c+d), \qquad
 t(a+ib) = ta + itb.
\]

Vogliamo estendere la moltiplicazione a tutte le coppie di numeri complessi.
Imponendo (arbitrariamente) che valga $i\cdot i = -1$ e che rimanga
valida la proprietà distributiva si ottiene
questa definizione:
\[
   (a+ib) \cdot (c+id) = (ac-bd) + i(ad+bc).
\]

Si può verificare che questa moltiplicazione estende quella "scalare" definita
in precedenza. Inoltre l'insieme $\CC$ equipaggiato delle due operazioni di
addizione e moltiplicazione risulta essere un campo.

Osserviamo che su $\CC$ non si definisce una operazione d'ordine perché
in effetti non è possibile definire un ordine "compatibile" con le operazioni
appena definite.

Su $\CC$ definiamo delle ulteriori operazioni.
Il \emph{coniugato} di un numero complesso $z=x+iy$ è il numero
$\bar z = x - iy$. Geometricamente l'operazione di coniugio è una simmetria
rispetto alla retta reale. I numeri reali sono in effetti punti fissi del
coniugio (il coniugato di un numero reale è il numero stesso).
Osserviamo che si ha
\[
z \cdot \bar z = (x+iy)(x-iy) = x^2-i^2y^2 = x^2+y^2.
\]

Il \emph{modulo} di un numero complesso $z=x+iy$
è il numero reale $ \abs{z} = \sqrt{z\cdot\bar z} = \sqrt{x^2+y^2}$.
Geometricamente tale quantità rappresenta la distanza del punto $z$
dal punto $0$ e quindi la distanza tra due numeri complessi $z$ e
$w$ si potrà rappresentare con $\abs{z-w}$.

Osserviamo che se $x\in \RR$ si ha
\[
\abs{x}
= \sqrt{x^2}
=
\begin{cases}
  x &\text{se $x\ge 0$}\\
  -x &\text{se $x<0$}.
\end{cases}
\]
In tal caso $\abs{x}$ si chiama \emph{valore assoluto}.

Il modulo di un numero complesso soddisfa le seguenti proprietà
\begin{enumerate}
\item $\abs{\abs{z}} = \abs{z}$ (idempotenza),
\item $\abs{-z} = \abs{z}$, $\abs{\bar z}$ (simmetria),
\item $\abs{z+w} \le \abs{z}+\abs{w}$ (convessità),
\item $\abs{z-w} \le \abs{z-v} + \abs{v-w}$ (disuguaglianza triangolare),
\item $\abs{z\cdot w} = \abs{z}\cdot\abs{w}$ (omogenità).
\end{enumerate}

Possiamo a questo punto trovare una utile formula per calcolare
il reciproco di un numero complesso. Essendo infatti
$z\cdot \bar z = \abs{z}^2$ si osserva che
\[
  \frac{1}{z} = \frac{\bar z}{\abs{z}^2}.
\]

\section{rappresentazione polare dei numeri complessi}

I numeri complessi di modulo uno vengono chiamati \emph{unitari}.
Geometricamente i numeri complessi unitari sono i punti della circonferenza
unitaria centrata nell'origine del piano complesso.
I prodotti e i
reciproci dei numeri complessi unitari sono anch'essi unitari,
risulta quindi che tali numeri formano un \emph{sottogruppo moltiplicativo}%
\footnote{
Un \emph{gruppo} è un insieme su cui è definita una operazione
(spesso denotata con il simbolo della moltiplicazione) che sia associativa,
che abbia elemento neutro e tale che ogni elemento abbia un inverso.
}
del gruppo dei numeri complessi.
Tra essi compaiono (come intersezioni con gli assi reale e immaginario)
i numeri $1, i, -i, -1$ che rappresentano, rispettivamente, gli angoli:
nullo, retto, piatto e ancora retto.

I numeri complessi unitari possono essere utilizzati per rappresentare gli
angoli geometrici. Se $\theta$ è unitario, cioè $\abs{\theta}=1$, possiamo
pensare che $\theta$ rappresenti l'angolo con vertice nell'origine, delimitato
dall'asse dei reali positivi e dalla semiretta
uscente da $0$ e passante per $\theta$.

Ogni numero complesso $z$ potrà essere scritto nella forma
\[
  z = \rho \theta
\]
con $\rho>0$ (sottointeso $\rho \in \RR$ visto che il confronto $<$ non
ha senso sui complessi) e $\theta$ unitario.
L'angolo corrispondente a $\theta$ viene chiamato \emph{argomento}
del numero complesso $z$ e si denota a volte con $\arg z$.
Basta infatti
definire $\rho = \abs{z}$ e $\theta = z / \abs{z}$ (se $z\neq 0$, altrimenti
si potrà scegliere arbitrariamente $\theta=0$).
La rappresentazione di un numero complesso tramite modulo e argomento
si chiama \myemph{rappresentazione polare}.
Si contrappone alla
\myemph{rappresentazione cartesiana} $z=x+iy$ in cui
vengono evidenziate le coordinate cartesiane del punto $z$
rispetto ai due assi reale e immaginario.

Consideriamo la funzione $R_\theta\colon \CC \to \CC$,
$R_\theta(z) = \theta\cdot z$.
Dalla proprietà distributiva della moltiplicazione complessa risulta
immediato che tale funzione è lineare (stiamo pensando a $\CC$ come
spazio vettoriale bidimensionale su $\RR$).
Osserviamo che $R_\theta(1) = \theta$ cioe $R_\theta$ agisce sul punto $1$
ruotandolo di un angolo $\theta$ sul piano complesso.
Se $\theta=x+ i y$ allora $R_\theta(i) = \theta\cdot i = y - ix$ e si osserva
che il punto di coordinate $(-x, y)$ di nuovo
non è altro che il ruotato di un angolo
$\theta$ del punto $i$. Visto che la rotazione è una funzione lineare e coincide
su una base dello spazio con $R_\theta$ scopriamo che $R_\theta$ deve coincidere con
la rotazione di un angolo $\theta$ su tutti i punti di $\CC$.

In particolare se $\alpha$ e $\beta$ sono unitari, il loro prodotto
$\alpha\beta$ corrisponde alla rotazione di $\beta$ di un angolo $\alpha$
ovvero alla somma degli angoli $\alpha$ e $\beta$.
Dunque la moltiplicazione complessa sui numeri unitari rappresenta la somma
degli angoli corrispondenti. Si può usare questa interpretazione
per dare significato geometrico alle identità algebriche: $i^2=-1$, $(-1)^2 = 1$,
$(-i)^2 = -1$.

Possiamo allora più in generale intepretare il prodotto di due numeri complessi
$z\cdot w$. Se $z\neq 0$ possiamo scrivere $z = \abs{z} \theta$ con
$\theta= z/\abs{z}$ unitario, cosicché:
\[
  z \cdot w = \abs{z} R_\theta(w).
\]
Si capisce quindi che il numero complesso $z\cdot w$ si ottiene ruotando
$w$ dell'angolo identificato da $z$ con l'asse dei reali positivi, e quindi
riscalando il punto ottenuto di un fattore $\abs{z}$.
In pratica il numero $z\cdot w$ in coordinate polari è quel numero
complesso il cui argomento è la somma degli argomenti dei due fattori
e il cui modulo è il prodotto dei moduli.


% \bibliographystyle{plain}
% \bibliography{biblio}

\end{document}
